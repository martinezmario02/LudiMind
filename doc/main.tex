\documentclass[12pt]{article}
\usepackage[margin=2.54cm]{geometry}
\usepackage{graphicx}
\usepackage[utf8]{inputenc}
\usepackage{csquotes}
\usepackage{amsmath}
\graphicspath{ {./imgs/} }
\usepackage[spanish]{babel}
\usepackage{hyperref}
\hypersetup{
  colorlinks=true,
  linkcolor=black,
  urlcolor=blue
}
\usepackage{url}
\usepackage{float}
\usepackage{enumitem}
\usepackage{comment}
\usepackage{wasysym}
\usepackage{multirow}
\usepackage[utf8]{inputenc}
\usepackage[usenames]{color}
\usepackage[document]{ragged2e}
\usepackage[table]{xcolor}
\usepackage{colortbl}
\definecolor{lightgray}{gray}{0.95}
\definecolor{black}{rgb}{0, 0, 0}
\renewcommand{\arraystretch}{1.3}
\arrayrulecolor{black}
\setlength{\arrayrulewidth}{0.8pt}
\usepackage{tabularx}
\usepackage{booktabs} 
\usepackage{longtable}
\usepackage{pdflscape}

\usepackage[backend=biber,style=apa]{biblatex} % Citación APA
\addbibresource{bibliografia.bib}
\usepackage{csquotes}
\usepackage{setspace}
\onehalfspacing
\usepackage{titlesec}
\usepackage{makecell}

\usepackage{cleveref}
\crefformat{figure}{(#2Fig.~#1#3)}
\Crefformat{figure}{(#2Fig.~#1#3)}
\newcommand{\tabref}[1]{Cuadro~\ref{#1}}

\setcounter{secnumdepth}{4}
\setcounter{tocdepth}{4} 
\makeatletter
\makeatletter
\renewcommand\paragraph{\@startsection{paragraph}{4}{0mm}%
  {3.25ex plus 1ex minus .2ex}%
  {1.5ex plus .2ex}%
  {\normalfont\normalsize\bfseries}}
\makeatother

\begin{document}

% Portada
\begin{titlepage}
    \centering
    
    % Logo UGR arriba
    \includegraphics[width=0.6\textwidth]{imgs/UGR-Logo.png}
    
    \vspace{0.5cm}

    {\LARGE TRABAJO DE FIN DE MÁSTER \par}

    \vspace{0.5cm}
    
    {\Large Máster Universitario en Ingeniería Informática\\
    Universidad de Granada \par}
    
    \vspace{0.7cm}
    
    {\Huge \textbf{LudiMind} \par}
    
    {\LARGE \textit{Sistema de estimulación cognitiva mediante juegos interactivos para estudiantes con TDAH} \par}
    
    \vspace{0.7cm}
    
    % Logo ETSIIT abajo
    \includegraphics[width=0.35\textwidth]{imgs/ETSIIT-logo.png}
    
    {\Large \textbf{Autor} \par}
    {\Large Mario Martínez Sánchez \par}

    \vspace{0.4cm}
    
    {\Large \textbf{Tutores} \par}
    {\Large María José Rodríguez Fórtiz \par}
    {\Large Dulce Romero Ayuso \par}
    
    \vfill
    
\end{titlepage}
\newpage
\justifying

% Páginas informativas
\newpage
\thispagestyle{empty}
\null
\newpage

\begin{center}
    \textbf{LudiMind: Sistema de estimulación cognitiva mediante juegos interactivos para niños con TDAH}
    
    Mario Martínez Sánchez
\end{center}
\vspace{2cm}

\textbf{Resumen}

Las personas que tienen Trastorno por Déficit de Atención e Hiperactividad (TDAH) enfrentan diariamente problemas relacionados con la autorregulación, la impulsividad, la concentración y la planificación. 
En el caso de los estudiantes, estos problemas pueden afectar a su desarrollo académico, social y emocional, especialmente si no se aplican estrategias de intervención adecuadas.

Con el propósito de proporcionar una aplicación lúdica y educativa que favorezca la mejora de estas funciones ejecutivas, en este Trabajo de Fin de Máster se ha creado una aplicación digital de minijuegos educativos enfocada en niños y niñas con TDAH. 
El proyecto tiene como objetivo estimular habilidades fundamentales como la memoria de trabajo o la planificación, utilizando dinámicas motivadoras, refuerzo positivo y un diseño adaptado a sus necesidades cognitivas.

Para el desarrollo de la aplicación, se ha utilizado React.js para el frontend y Node.js con Express para el backend, asegurando una experiencia de usuario fluida y accesible desde múltiples dispositivos. Además, se ha levantado en contendores Docker para facilitar su despliegue y mantenimiento.

Cada juego está diseñado para desarrollar una competencia específica, brindando retroalimentación inmediata y una experiencia atractiva que promueve la autonomía, el pensamiento crítico y la perseverancia. Además, se incluye un sistema de seguimiento del progreso y recompensas para mantener el compromiso del usuario.

Este enfoque integra la evidencia científica sobre la intervención en TDAH con los principios del aprendizaje basado en el juego, ofreciendo un entorno accesible y efectivo que complementa otras terapias o apoyos educativos.
\\

\textbf{Palabras clave}

Trastorno por Déficit de Atención e Hiperactividad, Funciones ejecutivas, Juegos educativos, Atención, Motivación, aplicación digital, Memoria de trabajo, Planificación, Autorregulación emocional, Resolución de conflictos, Estudiantes, Usabilidad, Gamificación, Refuerzo positivo, Evaluación, Intervención psicoeducativa, Neurodesarrollo, Aprendizaje basado en juegos, Terapias asistidas por ordenador, Tecnología educativa.

\newpage
\thispagestyle{empty}
\null
\newpage

\begin{center}
    \textbf{LudiMind: Cognitive stimulation system through interactive games for children with ADHD}
    
    Mario Martínez Sánchez
\end{center}
\vspace{2cm}

\textbf{Abstract}

People with Attention Deficit Hyperactivity Disorder (ADHD) face daily challenges related to self-regulation, impulsivity, concentration, and planning. 
For students, these problems can affect their academic, social, and emotional development, especially if appropriate intervention strategies are not implemented.

With the aim of providing a fun and educational tool that favors the improvement of these executive functions, this Final Master Project created a digital platform of educational minigames focused on children with ADHD. The application aims to stimulate fundamental skills such as working memory and planning, using motivating dynamics, positive reinforcement, and a design adapted to their cognitive needs.

The platform was developed using React.js for the frontend and Node.js with Express for the backend, ensuring a fluid user experience accessible across multiple devices. It was also built in Docker containers for easy deployment and maintenance.

Each game is designed to develop a specific skill, providing immediate feedback and an engaging experience that promotes autonomy, critical thinking, and perseverance. In addition, a progress tracking and rewards system are included to maintain user engagement.

This approach integrates scientific evidence on ADHD intervention with the principles of play-based learning, offering an accessible and effective environment that complements other therapies or educational supports.
\\

\textbf{Keywords}

Attention Deficit Hyperactivity Disorder, Executive Functions, Educational Games, Attention, Motivation, Digital Platform, Working Memory, Planning, Emotional Self-Regulation, Conflict Resolution, Students, Usability, Gamification, Positive Reinforcement, Assessment, Psychoeducational Intervention, Neurodevelopmental Disorders, Game-Based Learning, Computer-Assisted Therapy, Educational Technology.
\newpage

\newpage
\thispagestyle{empty}
\null
\newpage

Yo, \textbf{Mario Martínez Sánchez}, alumno de la titulación Máster en Ingeniería Informática de la \textbf{Escuela Técnica Superior de  Ingenierías  Informática y de Telecomunicación de la Universidad de Granada}, con DNI 23310000Y, autorizo la ubicación de la siguiente copia de mi Trabajo Fin de Máster en la biblioteca del centro para que pueda ser consultada por las personas que lo deseen.

\vspace{4cm}

Fdo: Mario Martínez Sánchez

\vspace{2cm}

\begin{flushright}
    Granada, a 01 de septiembre de 2025.
\end{flushright}
\newpage

\newpage
\thispagestyle{empty}
\null
\newpage

\textbf{María José Rodríguez Fortiz}, Profesora del

Departamento de Lenguajes y Sistemas 

Informáticos de la Universidad de Granada.
\\

\textbf{Dulce Romero Ayuso}, Profesora 

del Departamento de Fisioterapia de 

la Universidad de Granada.

\vspace{2cm}

\textbf{Informan:}

\vspace{1cm}

Que el presente trabajo, titulado \textbf{LudiMind: Sistema de estimulación cognitiva mediante juegos interactivos para niños con TDAH}, ha sido realizado bajo su supervisión por \textbf{Mario Martínez Sánchez}, y autorizamos la defensa de dicho trabajo ante el tribunal que corresponda.
\\

Y para que conste, expiden y firman el presente informe en Granada a 1 de septiembre de 2025.

\vspace{2cm}
\textbf{Las directoras:}

\vspace{4cm}
\begin{center}
    \textbf{María José Rodríguez Fortiz \hspace{2cm}
    Dulce Romero Ayuso}
\end{center}

\newpage
\thispagestyle{empty}
\null
\newpage

\subsection*{Agradecimientos}
\newpage

\newpage
\thispagestyle{empty}
\null
\newpage

% Índices
\newpage
\tableofcontents
\newpage
\listoffigures
\newpage
\listoftables

% Secciones
\newpage
\section{Introducción}

\subsection{Motivación}
El TDAH, también conocido como Trastorno por Déficit de Atención e Hiperactividad, es una afección crónica que experimentan millones de niños y niñas, continuando en muchos casos en la edad adulta. 
La prevalencia mundial en la etapa infantil se encuentra entre el 5 y el 7\%, siendo el trastorno que produce el mayor número de consultas en pediatría \parencite{llanos2019tdah}. 
Entre los principales síntomas que manifiesta este colectivo encontramos la falta de atención y una conducta hiperactiva e impulsiva \parencite{leffa2022adhd}. 

A pesar de la existencia de recursos educativos generales, pocos están específicamente diseñados para adaptarse a las características cognitivas y emocionales del TDAH. 
En este contexto, una aplicación basada en juegos interactivos puede convertirse en una herramienta eficaz y motivadora, promoviendo el aprendizaje a través del juego y favoreciendo el desarrollo de las áreas clave afectadas por el TDAH.

Por otro lado, a pesar del gran número de personas que lo experimentan, la población actual no está totalmente concienciada del problema que esto supone para este colectivo. 
De esta forma, escogí este proyecto no solo en busca de ayudar a estas personas a mejorar diversas habilidades, sino que también tratando de buscar una mayor conciencia y comprensión de la población sobre el TDAH, promoviendo un enfoque más inclusivo hacia las personas que viven esta condición diariamente.

Además, este proyecto resulta muy significativo para mí, al tener un familiar cercano con TDAH, lo que me ha permitido también observar personalmente las dificultades cotidianas a las que se enfrentan dichas personas, tanto en el ámbito escolar como en su vida diaria. 
Esto se vio reforzado por mi experiencia previa trabajando con niños y  niñas con TDAH durante el desarrollo de mi Trabajo de Fin de Grado, donde pude profundizar en sus necesidades al desarrollar otra plataforma para este colectivo. 
Este proyecto consistió en una aplicación móvil basada en una agenda virtual en la que los estudiantes con TDAH pudieran apuntar sus tareas diarias, organizándose semanalmente y recibiendo asesoramiento para poder mejorar su estilo de vida.

Así pues, con este trabajo de fín de máster aspiro a afianzar mis conocimientos en diseño centrado en el usuario y la accesibilidad, al tiempo que contribuyo a la inclusión y comprensión del TDAH en la sociedad. 
Para ello, la plataforma contará con varios minijuegos con distintos niveles, los cuales fomenten el desarrollo de habilidades como la memoria, la organización, la resolución de conflictos o la gestión del tiempo.

\newpage

\subsection{Objetivos del proyecto}
Este proyecto se plantea como una herramienta lúdica y educativa destinada a estudiantes con TDAH, con el fin de apoyar el desarrollo de sus habilidades cognitivas, tales como la atención, la memoria de trabajo y la supervisión de la acción.

Entre los objetivos de este proyecto destacan:
\begin{enumerate}
    \item Revisar aplicaciones similares a la planteada para la extracción de características positivas y negativas.
    \item Revisar tecnologías potenciales a ser utilizadas para poder comparar y decidir sobre las más idónea.
    \item Revisar trabajos de investigación relacionados
    \item Desarrollar una plataforma con interfaz accesible, atractiva y adaptada a la población con TDAH.
    \item Diseñar minijuegos con distintas mecánicas que trabajen áreas concretas como la memoria, la organización o la resolución de conflictos sociales.
    \item Incorporar elementos de gamificación que motiven a los usuarios mediante misiones y recompensas.
    \item Implementar un apartado de retroalimentación que permita al usuario y a sus tutores consultar su evolución de forma visual y sencilla.
    \item Evaluar la herramienta con perfiles representativos del público objetivo, ya sean estudiantes, familias o profesores.
\end{enumerate}

Con ello, se espera ofrecer un recurso innovador y totalmente funcional que se adapte a las necesidades de este público y contribuya de forma lúdica al desarrollo de los niños y niñas con TDAH.
\newpage

\subsection{Estructura del documento}

En este apartado se describe la organización del presente documento, detallando de forma resumida el contenido que se abordará en cada sección. La finalidad de esta estructura es ofrecer al lector una visión clara y ordenada del desarrollo del trabajo, facilitando la localización de la información y la comprensión.

La memoria se compone de las siguientes secciones principales:

\begin{itemize}
    \item \textbf{Introducción}: presenta la motivación del proyecto, sus objetivos y la estructura general del documento.
    \item \textbf{Estado del arte}: expone artículos científicos previos y el análisis de soluciones similares, así como la identificación de las necesidades del usuario.
    \item \textbf{Análisis}: recoge las historias de usuario, la especificación de requisitos y la representación gráfica mediante diagramas de casos de uso.
    \item \textbf{Propuesta}: define la solución planteada, las herramientas de desarrollo seleccionadas, la planificación de las fases del proyecto y su estructura.
    \item \textbf{Mecánica de los juegos}: describe en detalle el funcionamiento de cada uno de los juegos que integran la plataforma.
    \item \textbf{Bibliografía}: relación de todas las fuentes consultadas para la elaboración de la memoria.
\end{itemize}


\newpage
\section{Estado del arte}

\subsection{Características del TDAH}
El trastorno por déficit de atención con hiperactividad (TDAH) es un trastorno del neurodesarrollo \parencite{brown2007}.
Se diagnostica frecuentemente en la infancia y suele persistir en la edad adulta. 
Este trastorno se caracteriza por la inatención (fallos en la regulación de la atención, distracción moderada a grave, períodos de atención breve o excesiva), 
hiperactividad (inquietud principalmente mental, no en todos los subtipos) 
y comportamiento impulsivo (inestabilidad emocional y conductas impulsivas, incluyendo la inquietud motora) que produce problemas en múltiples áreas de funcionamiento, dificultando el desarrollo social, emocional y cognitivo de la persona que lo tiene \parencite{marcelo2013}.
Una característica común, y que parece ser contraria al nombre del trastorno, es la capacidad de hiperfoco o hiperconcentración, debido a que las personas con TDAH poseen una atención ligada a factores motivacionales, no conscientes, y pueden prestar atención excesiva a aquello que les proporciona recompensa inmediata \parencite{kooij2019}.
Podemos encontrar 3 tipos de TDAH según predomine la falta de atención, la hiperactividad/impulsividad o una combinación de ambas \parencite{msd_tdah_profesional}.

Los estudiantes con TDAH presentan una gran variedad de dificultades en diversas áreas del desarrollo cognitivo, emocional y social. 
Esto afecta de forma considerable a su vida cotidiana y su desempeño escolar \parencite{fundacioncadah}. 
Entre las áreas afectadas de forma más frecuente se encuentran la memoria de trabajo, la planificación, la autorregulación emocional, la resolución de conflictos sociales y la gestión del tiempo.

Todas estas dificultades desembocan en problemas como el olvido de instrucciones, la dificultad para el inicio de tareas, o la reacción de forma impulsiva ante la frustración. 
Así pues, estos niños y niñas pueden llegar a experimentar rechazo social o conflictos por su impulsividad y dificultad para interpretar normas sociales.

Además, la normativa educativa en España destaca la gran utilidad de herramientas tecnológicas dedicadas a la atención de las necesidades específicas del alumnado. 
De acuerdo con la Ley Orgánica 3/2020 (LOMLOE), las administraciones educativas deben asegurar los recursos necesarios para que los estudiantes que requieran una atención educativa diferente a la ordinaria, por presentar necesidades educativas especiales (en lo que incluye el TDAH), para que puedan alcanzar el máximo desarrollo posible de sus capacidades personales y, a ser posible, lograr los objetivos establecidos con carácter general para todo el alumnado \parencite{lomloe_boe}.
\\

Por otro lado, asociaciones como la Federación Española de Asociaciones de TDAH (FEAADAH) han destacado la importancia de desarrollar y usar herramientas tecnológicas adaptadas que fomenten una educación inclusiva y equitativa, contribuyendo a la atención de la diversidad de forma más eficaz. 
Este enfoque coincide con el planteamiento de la plataforma DiversiAE, de la que FEAADAH forma parte \parencite{feaadah_diversiae}.

Así pues, se ha demostrado que las personas con TDAH responden de forma positiva cuando se les ofrecen entornos estructurados, rutinas fijas y dinámicas motivadoras \parencite{tech_interventions_tdah}. 
Diversos estudios recientes también destacan que el uso de juegos serios y tecnologías interactivas puede tener un impacto positivo en la mejora de la atención, la memoria de trabajo y la autorregulación en población infantil con TDAH \parencite{doulou2025}. 
Es por esto que los juegos interactivos propuestos podrían ser una herramienta eficaz para estimular habilidades cognitivas y emocionales, favoreciendo el aprendizaje mediante el juego.

De este modo, se plantea como proyecto la implementación de una plataforma de minijuegos que pueda incorporarse al proyecto \textit{DiverAccion}, el cual pretende mejorar habilidades como la memoria, planificación, organización y gestión del tiempo en adolescentes con diagnóstico de TDAH, entre 9 y 17 años. 
Este proyecto constará de un sistema de tele-rehabilitación en colaboración con los principales agentes implicados en el proceso de intervención en las actividades diarias y académicas \parencite{diveraccion}.

\subsection{Juegos serios para personas con NEAE}
Los juegos serios son aplicaciones diseñadas con un fin educativo o terapéutico que combinan elementos lúdicos con objetivos específicos de aprendizaje o mejora de habilidades.
En el contexto de las personas con Necesidades Específicas de Apoyo Educativo (NEAE), los juegos serios pueden ser una herramienta eficaz para abordar diversas áreas de desarrollo, como la cognición, la motricidad, la socialización y la autorregulación emocional.
Estos juegos permiten a los usuarios interactuar con entornos controlados donde puedan practicar habilidades sin la presión de un entorno clínico.
Esto puede facilitar el ritmo de aprendizaje y la retroalimentación inmediata \parencite{neurekalab2022}.

En el caso específico de personas con TDAH, estos juegos serios han mostrado ser muy efectivos para mejorar la atención sostenida y la memoria de trabajo.
Un estudio realizado en Cataluña implementó un programa que contaba con diversos juegos serios, incluyendo actividades como selección de estímulos o laberintos.
Los resultados obtenidos indicaron mejoras significativas en la atención sostenida y la memoria de trabajo en niños y niñas con TDAH tras la intervención \parencite{redalyc2019}.

También se han desarrollado juegos destinados a la evaluación de características de este trastorno, investigando para el desarrollo de herramientas que ayudaran en el diagnóstico del TDAH mediante la evaluación de funciones ejecutivas \parencite{pmc2014};
así como la mejora de la adherencia al tratamiento al hacer las intervenciones más atractivas y motivadoras.

Así pues, es crucial que los juegos serios sean diseñados incorporando recompensas y elementos de gamificación que fomenten la participación activa de los usuarios; así como niveles progresivos y retroalimentación positiva para mantener el interés. Además, se deben diseñar actividades que trabajen áreas clave afectadas por el TDAH, como la memoria, la organización o la regulación emocional.

\subsection{Propuestas similares}
Antes de comenzar con el desarrollo de la aplicación, se ha realizado una búsqueda de aplicaciones que pudieran servir de guía, extrayendo tanto los puntos fuertes como los débiles.

\subsubsection{Respira, piensa, actúa}
Este primer juego \parencite{rpa} trata de ayudar a niños y niñas a aprender cómo resolver problemas siguiendo el siguiente patrón: respirar, pensar y actuar. 
Para ello, los usuarios deben tocar la pantalla para ayudar al monstruo a respirar profundamente, pensar en un plan y, finalmente, actuar de forma adecuada. 
Además, incorpora algunos juegos, como la posibilidad de explotar burbujas para ayudar al monstruo.
Esto permite al usuario practicar la respiración profunda, una técnica de autorregulación emocional que puede ser beneficiosa para personas con TDAH.

\begin{figure}[H]
    \centering
    \includegraphics[width=0.5\linewidth]{imgs/app_rpa.png}
    \caption{Respira, piensa, actúa}
\end{figure}

Este juego fomenta la autorregulación emocional en la población de menor edad, incorporando una narrativa guiada por personajes, así como la resolución de diversas situaciones. 
A través de diversas escenas cotidianas, el estudiante ayuda al personaje a identificar cómo se siente, a calmarse utilizando una técnica de respiración guiada y a buscar soluciones. 
Un ejemplo de esto es una situación en la que el personaje no puede abrocharse el abrigo.

\begin{figure}[H]
    \centering
    \includegraphics[width=0.5\linewidth]{imgs/app_rpa_burbujas.jpg}
    \caption{Minijuego de burbujas}
\end{figure}

No obstante, la aplicación a desarrollar en este proyecto debe estar adaptada a un perfil algo más maduro (entre 9 y 17 años aproximadamente) y con necesidades específicas. 
Además, nuestro proyecto trata de mejorar la autonomía y el pensamiento crítico mediante la toma de decisiones y retos, mientras que este otro juego sigue una narrativa guiada (pulsar para avanzar). 
Aun así, la estructura narrativa y el enfoque emocional de esta aplicación pueden servir como inspiración para diseñar experiencias similares en algunos minijuegos del proyecto.

\subsubsection{Lumosity - Entrenador Cerebral}
Esta otra aplicación \parencite{lumosity} consta de una gran variedad de juegos para ejercitar la memoria y entrenar el cerebro de forma interactiva y amena. 
Estos juegos están divididos en función del área a entrenar, tal y como se busca con nuestro proyecto.

\begin{figure}[H]
    \centering
    \includegraphics[width=0.5\linewidth]{imgs/app_lumosity.png}
    \caption{Lumosity - Entrenador Cerebral}
\end{figure}

A pesar de no estar diseñada específicamente para el TDAH, trabaja áreas como la memoria de trabajo y la planificación, aspectos clave a mejorar en el caso de las personas con TDAH, ya que están directamente relacionados con su capacidad para organizar tareas, seguir instrucciones y mantener la información relevante en mente durante la realización de una actividad. 
En este sentido, algunos juegos de la plataforma \textit{Lumosity} como \textit{Memory Matrix} estimulan la retención activa de patrones o secuencias visuales, reforzando la memoria de trabajo. 
Para ello, el usuario tiene que memorizar qué celdas de una matriz están coloreadas para, posteriormente, seleccionarlas.

\begin{figure}[H]
    \centering
    \includegraphics[width=0.5\linewidth]{imgs/app_lumosity_mm.jpg}
    \caption{Minijuego Memory Matrix}
\end{figure}

Por otro lado, juegos como Pirate Passage, que requiere planificar rutas óptimas para alcanzar un objetivo, fomentan habilidades de planificación y toma de decisiones. 

\begin{figure}[H]
    \centering
    \includegraphics[width=0.5\linewidth]{imgs/app_lumosity_ps.png}
    \caption{Minijuego Pirate Passage}
\end{figure}

Además, contiene una página con estadísticas que muestra el desempeño del usuario, lo cual puede resultar muy útil para personas con TDAH, ya que de este modo, pueden recibir retroalimentación acerca de sus logros y aspectos a mejorar. 

Sin embargo, aunque estos juegos se adaptan al nivel de los usuarios, algunos de ellos pueden llegar a resultar algo complejos cuando el usuario va avanzando, pues están realizados para un intervalo de edad mayor. 
Nuestra plataforma, por el contrario, estará centrada en un público más joven e inexperto, por lo que los juegos que se buscan tendrán una dificultad algo menor.
\newpage

\subsubsection{Cognifit - Test y Juegos}
Esta tercera aplicación \parencite{cognifit} es una plataforma de estimulación cognitiva basada en la neuropsicología y en la evaluación del rendimiento, permitiendo tanto la estimulación como el seguimiento personalizado.

\begin{figure}[H]
    \centering
    \includegraphics[width=0.5\linewidth]{imgs/app_cognifit.jpg}
    \caption{Cognifit - Test y Juegos}
\end{figure}

Una de sus principales fortalezas es la posibilidad de crear programas adaptados a perfiles específicos como el TDAH. 
Para la creación de perfiles, el usuario es expuesto a una serie de preguntas sobre sus hábitos, estilo de vida y dificultades cognitivas; obteniendo un perfil cognitivo personalizado.
Además, incluye juegos interactivos diseñados para fortalecer áreas cognitivas comúnmente afectadas en este trastorno, como la memoria de trabajo, la atención sostenida o la planificación.

Esta aplicación destaca por su gran cantidad de artículos científicos que avalan su eficacia en la mejora de diversas funciones cognitivas, mejorando significativamente la memoria de trabajo, realizando un gran entrenamiento cognitivo interactivo y gamificado; y evaluando y adaptando los ejercicios a nivel de usuario \parencite{shatil2014}.
Por ejemplo, para la memoria de trabajo, \textit{CogniFit} incluye tareas que requieren retener y manipular información por breves períodos, así como vídeos explicativos para cada área, lo que resulta útil en intervenciones orientadas a la mejora del rendimiento académico. 
A continuación, se muestra un minijuego en el que la plataforma muestra un número al usuario durante un segundo y este tiene que recordarlo y completar la ecuación disparando al objeto correcto.

\begin{figure}[H]
    \centering
    \includegraphics[width=0.5\linewidth]{imgs/app_cognifit_memory.jpg}
    \caption{Minijuego de memoria}
\end{figure}

Además, la plataforma puede ser utilizada tanto en entornos clínicos como en el hogar o el contexto escolar, y permite la supervisión por parte de profesionales, educadores o familias.

En definitiva, \textit{CogniFit} representa una herramienta digital útil para trabajar habilidades cognitivas específicas de forma lúdica y motivadora, algo destacable para el TDAH, donde el componente motivacional y la retroalimentación inmediata son fundamentales.

\subsection{Comparativa de aplicaciones}
Tras el análisis de otras aplicaciones, se ha realizado un estudio de aspectos clave como las competencias trabajadas en cada una de estas aplicaciones, el precio de uso,  el sistema operativo requerido o la edad a la que está dirigida. Con esta información, se han obtenido las siguientes tablas resumen:

\begin{table}[H]
\centering
\begin{tabularx}{\textwidth}{|X|X|X|X|}
    \hline
    \textbf{Competencia} & \textbf{Respira, Piensa, Actúa} & \textbf{Lumosity} & \textbf{CogniFit}\\
    \hline
    Autorregulación emocional         & x &   &   \\
    \hline
    Resolución de conflictos sociales & x &   &   \\
    \hline
    Identificación emocional          & x &   &   \\
    \hline
    Memoria de trabajo                &   & x & x \\
    \hline
    Organización de objetos/tareas    &   &   & x \\
    \hline
    Planificación / anticipación      &   & x & x \\
    \hline
    Control inhibitorio / impulsos    & x & x & x \\
    \hline
    Gestión del tiempo                &   & x &   \\
    \hline
    Motivación por refuerzo           & x & x & x \\
    \hline
\end{tabularx}
\caption{Comparativa entre competencias de aplicaciones}
\end{table}

\begin{table}[H]
\centering
\begin{tabularx}{\textwidth}{|X|X|X|X|}
    \hline
    \textbf{Característica} & \textbf{Respira, Piensa, Actúa} & \textbf{Lumosity} & \textbf{CogniFit} \\
    \hline
    Precio & Gratuita & Gratis con limitación; suscripción premium para escoger juegos & Suscripción obligatoria \\
    \hline
    Sistema Operativo & iOS, Android & Web, iOS, Android & Web, iOS, Android \\
    \hline
    Edad recomendada & 2–5 años & $\ge$ 13 años & 9–17 años \\
    \hline
    Nivel de dificultad & Fijo & Adaptativo & Adaptativo \\
    \hline
    Perfil TDAH & No & No & Sí \\
    \hline
    Personalización & No & Sí & Sí \\
    \hline
    Estudios que evalúan su eficacia & No & Sí & Sí \\
    \hline
    Registro de usuarios y perfiles & No & Sí & Sí \\
    \hline
    Seguimiento del progreso & No & Sí & Sí \\
    \hline
    Refuerzos y recompensas & No & Sí & Sí \\
    \hline
\end{tabularx}
\caption{Comparativa de las aplicaciones según características}
\end{table}

\subsection{Artículos científicos relacionados}
Para el diseño de una aplicación dirigida a niños y niñas con TDAH, resulta esencial revisar la evidencia científica disponible sobre el diseño y la eficacia de las herramientas tecnológicas en este ámbito. 
Así pues, se destacan dos artículos recientes que ofrecen claros resultados positivos ante las intervenciones tecnológicas sobre distintas funciones cognitivas y conductuales afectadas por el TDAH.

En primer lugar, encontramos el artículo \textit{Meta-analysis of the efficacy of digital therapies in children with attention-deficit hyperactivity disorder} \parencite{tech_interventions_tdah}, cuyo objetivo es la evaluación de la efectividad de intervenciones digitales para mejorar los síntomas del TDAH en estudiantes. 
Este estudio sigue las directrices PRISMA y el Manual Cochrane \parencite{prisma}, lo cual indica que la revisión es sistemática y fiable; además de estar registrado en PROSPERO \parencite{prospero}, lo que añade también transparencia al proceso.

Para la evaluación se utilizan instrumentos que podrían ser útiles para la validación de la plataforma. 
Por un lado, se utiliza la \textit{Escala de Calificación del Trastorno por Déficit de Atención e Hiperactividad (ADHD-RS)} para medir la inatención e impulsividad del alumnado \parencite{adhdrs}. 
Para ello, se basa en los criterios del \textit{Manual Diagnóstico y Estadístico de los Trastornos Mentales, Cuarta Edición (DSM-IV)}, el cual contiene descripciones detalladas de los diferentes trastornos mentales, incluyendo criterios diagnósticos específicos para cada uno \parencite{dsmiv}. 
Por otro lado, también se utiliza la herramienta \textit{Evaluación Conductual de la Función Ejecutiva-2 (BRIEF-2)}, un cuestionario completado por padres o tutores que mide las funciones ejecutivas, evaluando los aspectos más cotidianos y conductuales \parencite{brief2}.

Además de los resultados sobre eficacia, este metaanálisis describe las características de las terapias digitales incluidas, que abarcan fundamentalmente programas de entrenamiento cognitivo asistidos por ordenador y videojuegos terapéuticos. 
Estas intervenciones se orientan a mejorar habilidades como la atención sostenida, la memoria de trabajo o la autorregulación conductual. 
Una de sus características principales es el uso de mecánicas de gamificación (niveles progresivos, retroalimentación inmediata y recompensas), lo que favorece la motivación y la adherencia en la población infantil con TDAH.

En segundo lugar, encontramos el artículo \textit{Effectiveness of Technology-Based Interventions for School-Age Children With Attention-Deficit/Hyperactivity Disorder: Systematic Review and Meta-Analysis of Randomized Controlled Trials} \parencite{meta_digital_adhd}, cuyo objetivo es la realización de una revisión sistemática de las intervenciones tecnológicas para el alumnado con TDAH, así como un metaanálisis de los resultados tras estas intervenciones. 
Este estudio también está registrado en PROSPERO y sigue las directrices PRISMA para garantizar el rigor metodológico.

En este segundo artículo, se utilizaron herramientas de evaluación como las escalas de calificación, los cuestionarios o las pruebas estandarizadas relacionadas con síntomas de inatención e impulsividad. 
Aunque el artículo no detalla los nombres exactos de las herramientas utilizadas en cada ECA, sí especifica que las medidas empleadas evaluaban aspectos como inhibición, memoria de trabajo, atención, planificación, metacognición y calidad de vida. 
Esta metodología puede servir como referencia para la selección de instrumentos validados en futuras investigaciones, como por ejemplo el ADHD-RS o el BRIEF-2, ya utilizados en otros estudios similares.

En este caso, se analizan distintos tipos de intervenciones tecnológicas aplicadas en ensayos clínicos aleatorizados: programas de entrenamiento cognitivo digital, neurofeedback, aplicaciones educativas interactivas y entornos de realidad virtual. 
Estas herramientas trabajan un abanico de competencias relacionadas con el TDAH, entre ellas la atención visual, el control de impulsos, la memoria de trabajo y la planificación. 
El estudio destaca el uso de plataformas gamificadas o entornos lúdicos, lo que potencia la implicación del alumnado y contribuye a la mejora en medidas clínicas y funcionales.

\newpage
\section{Propuesta}

\subsection{Definición de la propuesta}
Este proyecto recibe el nombre de \textit{LudiMind}, el cual surge de la combinación de la palabra \textit{ludus} (del latín, juego o actividad lúdica) y \textit{mind} (palabra inglesa que significa mente).
De esta forma, \textit{LudiMind} transmite la idea de una plataforma que utiliza el juego como vía para estimular la mente, favoreciendo el aprendizaje, la autorregulación emocional y el desarrollo de habilidades cognitivas en niños y niñas con necesidades específicas de apoyo educativo, más concretamente, con TDAH.
Para ello, se desarrollarán cuatro juegos diseñados específicamente para trabajar áreas clave afectadas por el TDAH, todos ellos con numerosos niveles cuya complejidad se verá incrementada a medida que el jugador avance.
Estos juegos son guiados por un avatar que plantea situaciones, preguntas o retos que invitan a la reflexión o resolución de problemas, adaptados a las competencias que se desean trabajar.

\begin{itemize}
    \item \textbf{Juego de memoria y funciones ejecutivas:} Se tratará de sumergir a los jugadores y jugadoras en una aventura digital donde deben utilizar la red de metro de una ciudad ficticia, mientras desarrollan funciones ejecutivas y aprenden estrategias de memorización.
    El jugador o jugadora deberá recordar la parada de origen y destino, proporcionándole una frase mnemotécnica a modo de ayuda.
    A medida que se avanzan los niveles, se incrementa la dificultad proporcionando frases menos precisas.
    \item \textbf{Juego de planificación:} El jugador o jugadora accederá a un espacio mágico (desván mágico) que contendrá una gran cantidad de objetos desordenados (escolares, personales, fantásticos). 
    El objetivo será organizar los objetos arrastrándolos a los compartimentos correctos según diferentes criteros que cambiarán según el nivel. 
    Estos compartimentos podrán reflejar categorías de objetos u otras formas de organización (por iniciales, por tipo de palabra). 
    Además, se plantea la posibilidad de añadir niveles de mayor dificultad, en los cuales se agrupen distintas tareas en compartimentos, los cuales reflejen el tiempo que se dedicará a dicha tarea. 
    Esto podría ayudarlos a desarrollar otro área clave como la gestión del tiempo.
    \item \textbf{Juego de autorregulación emocional:} Se tomará el rol de \textit{detective emocional} y se analizarán mensajes, escenas de redes sociales o conversaciones familiares. 
    Estas situaciones se mostrarán a modo de imagen, planteando una situación ficticia.
    Su objetivo será identificar emociones complejas en simulaciones de contextos sociales realistas y valorar su intensidad, mejorando la conciencia social y la empatía.
    \item \textbf{Juego de resolución de conflictos sociales:} Se mostrarán diferentes escenarios de conflicto social, teniendo que analizar la situación utilizando los pasos del \textit{semáforo emocional}, una estrategia visual y sencilla adaptada a la población con TDAH. 
    Estos escenario también se podrán mostrar a modo de imagen o texto (un avatar explicando la situación).
    Esto desembocará en una mejora en la toma de decisiones, la autorregulación emocional y las habilidades sociales.
\end{itemize}

La aplicación se diseñará con una interfaz accesible y adaptada a personas con TDAH de menor edad; incorporando retroalimentación positiva, refuerzos motivadores y una curva de dificultad gradual entre los diferentes niveles de cada juego, garantizando que el aprendizaje no llegue a generar frustración.
Además, contará con una serie de misiones y recompensas para cada nivel, las cuales estarán disponibles desde el inicio. Estas recompensas podrán ser insignias o medallas virtuales, las cuales se irán desbloqueando a medida que se vayan completando las misiones. 
De esta forma, se busca fomentar la motivación intrínseca y el sentido de logro en los usuarios.

Esta propuesta busca no solo apoyar el desarrollo de habilidades cognitivas y emocionales; sino también mejorar la autonomía, la motivación y la autoestima de este colectivo, contribuyendo a una experiencia educativa más inclusiva y divertida.

\subsection{Metodologías y herramientas para la planificación}
Para llevar a cabo el desarrollo de la aplicación, se ha optado por utilizar una metodología ágil, concretamente \textbf{Scrum}, con el fin de facilitar una planificación flexible, iterativa y centrada en la mejora continua. Esto nos permite adaptarnos de forma dinámica a posibles cambios que puedan ir surgiendo mientras se prueba la aplicación.

Scrum estructura el desarrollo en ciclos de trabajo denominados \textit{sprints}, al final de los cuales se obtiene un incremento funcional del producto \parencite{proyectosagiles}.

Para la organización y seguimiento del proyecto se ha empleado la herramienta \textbf{Jira}, una plataforma especializada en la gestión de proyectos ágiles. Esto nos ha permitido la creación de un \textit{backlog} con las tareas a desarrollar, con su priorización y duración. Además, facilita la visualización de los estados de cada tarea, lo que hace más visual el estado del proyecto.

Gracias a la combinación de Scrum y Jira, ha sido posible mantener una planificación estructurada, con objetivos definidos por sprint, así como una organización clara de los recursos y tiempos destinados al desarrollo de los distintos minijuegos educativos de la aplicación.

En este caso, la realización del proyecto está dividida en iteraciones cada 3 semanas.

\subsection{Requisitos y análisis de la plataforma}
Tras el análisis de propuestas similares, se han extraído las características más apropiadas para el perfil del usuario objetivo, pero también se detectaron ciertos aspectos mejorables o poco adaptados. Estos elementos fueron replanteados en el diseño de esta plataforma con el objetivo de optimizar la experiencia del usuario.

Así, este proyecto no solo toma como referencia propuestas existentes, sino que también propone una evolución de las mismas, centrándose en áreas clave para las personas con TDAH.

\subsubsection{Product Backlog}

\paragraph{Listado completo}
Las historias de usuario constan de breves descripciones sobre la funcionalidad del sistema desde la perspectiva del usuario que utiliza la web. De este modo, se centran en las necesidades y objetivos de este usuario.

El siguiente listado cuenta con todas las historias de usuario de nuestra plataforma, asignándoles una estimación de esfuerzo y una prioridad. La estimación del esfuerzo está expresada en Puntos de Historia siguiendo la secuencia de Fibonacci (1, 2, 3, 5, 8),  mientras que la prioridad está medida siguiendo el método MoSCoW (Must, Should, Could, Won’t).
	
El método MoSCow es una técnica para priorizar tareas y funcionalidades en la gestión de proyectos, pudiendo asignar cualquiera de los siguientes valores:

\begin{itemize}
    \item \textbf{Must:} debe tener
    \item \textbf{Should:} debería tener
    \item \textbf{Could:} podría tener
    \item \textbf{Won’t:} no tendrá
\end{itemize}

\begin{table}[H]
\centering
\begin{tabularx}{\textwidth}{|X|p{0.6\textwidth}|X|X|X|}
\hline
\textbf{Ident.} & \textbf{Título} & \textbf{Est.} & \textbf{Prio.} & \textbf{Iter.} \\
\hline
HU.1 & Como usuario necesito registrarme. & 2 & M & 2 \\
\hline
HU.2 & Como usuario necesito iniciar sesión. & 3 & M & 2 \\
\hline
HU.3 & Como usuario necesito ver mi perfil. & 2 & C & 5 \\
\hline
HU.4 & Como usuario necesito modificar mi perfil. & 2 & C & 5 \\
\hline
HU.5 & Como usuario necesito consultar los juegos utilizados anteriormente. & 3 & S & 2 \\
\hline
HU.6 & Como usuario necesito consultar juegos nuevos. & 3 & M & 2 \\
\hline
HU.7 & Como usuario necesito consultar información sobre un juego concreto. & 2 & M & 3 \\
\hline
HU.8 & Como usuario necesito acceder a un tutorial de cada juego. & 7 & S & 6 \\
\hline
HU.9 & Como usuario necesito jugar al juego \textit{Metro de la memoria}. & 7 & M & 3 \\
\hline
HU.10 & Como usuario necesito jugar al juego \textit{Desván mágico}. & 7 & M & 4 \\
\hline
HU.11 & Como usuario necesito jugar al juego \textit{Detective emociones}. & 7 & M & 4 \\
\hline
HU.12 & Como usuario necesito jugar al juego \textit{Semáforo emocional}. & 7 & M & 5 \\
\hline
HU.13 & Como usuario necesito ver mi progreso en cada juego. & 3 & M & 3 \\
\hline
HU.14 & Como usuario necesito consultar mis misiones pendientes. & 3 & C & 6 \\
\hline
HU.15 & Como usuario necesito consultar mis misiones realizadas. & 3 & C & 6 \\
\hline
HU.16 & Como usuario necesito recibir recompensas al realizar misiones. & 5 & C & 6 \\
\hline
HU.17 & Como usuario necesito consultar las recompensas obtenidas. & 5 & C & 6\\
\hline
HU.18 & Como usuario necesito consultar la evolución de mis habilidades. & 5 & S & 5 \\
\hline
HU.19 & Como usuario necesito poder eliminar mi cuenta. & 1 & S & 5 \\
\hline
\end{tabularx}
\caption{Product Backlog}
\end{table}
\newpage

\paragraph{Tarjetas de Historias de Usuario}
Seguidamente, se muestran las tarjetas de las historias de usuario, las cuales permiten desarrollar cada historia de forma estructurada, especificando la información clave. Este formato facilita el seguimiento del avance del proyecto, la organización del trabajo en fases y la toma de decisiones en función de los objetivos planteados.

\begin{table}[H]
\centering
\begin{tabularx}{\textwidth}{|X|}
\hline
\textbf{HU.1 - Registro de usuarios}  \\
\hline
Como usuario, necesito registrarme en la plataforma introduciendo mis datos personales (nombre, fecha de nacimiento, correo electrónico y contraseña) para poder acceder a todas las funcionalidades disponibles y guardar mi progreso. 
\clearpage
\textbf{Estimación: }2
\clearpage
\textbf{Prioridad: }M
\\
\hline
\textbf{Pruebas de aceptación:}
\begin{itemize}
    \item Completar el formulario de registro y comprobar que se guarda correctamente.
    \item Intentar registrar un usuario con datos ya existentes y verificar que se muestra un error.
    \item Dejar campos obligatorios vacíos y verificar la validación del formulario.
\end{itemize}
 \\
\hline
\end{tabularx}
\caption{HU.1 - Registro de usuarios}
\end{table}

\begin{table}[H]
\centering
\begin{tabularx}{\textwidth}{|X|}
\hline
\textbf{HU.2 - Inicio de sesión}  \\
\hline
Como usuario, necesito iniciar sesión en la plataforma para acceder a los juegos disponibles.
\clearpage
\textbf{Estimación: }3
\clearpage
\textbf{Prioridad: }M
\\
\hline
\textbf{Pruebas de aceptación:}
\begin{itemize}
    \item Iniciar sesión con credenciales válidas y acceder correctamente.
    \item Probar credenciales incorrectas y verificar que se muestra un mensaje de error.
    \item Verificar que la sesión se mantiene activa.
\end{itemize}
 \\
\hline
\end{tabularx}
\caption{HU.2 - Inicio de sesión}
\end{table}

\begin{table}[H]
\centering
\begin{tabularx}{\textwidth}{|X|}
\hline
\textbf{HU.3 - Visualización de perfil}  \\
\hline
Como usuario, necesito consultar mis datos personales.
\clearpage
\textbf{Estimación: }2
\clearpage
\textbf{Prioridad: }C
\\
\hline
\textbf{Pruebas de aceptación:}
\begin{itemize}
    \item Acceder a la sección de perfil y visualizar los datos registrados.
\end{itemize}
 \\
\hline
\end{tabularx}
\caption{HU.3 - Visualización de perfil}
\end{table}

\begin{table}[H]
\centering
\begin{tabularx}{\textwidth}{|X|}
\hline
\textbf{HU.4 - Modificar perfil}  \\
\hline
Como usuario, necesito modificar mi perfil para actualizar mis datos personales cuando sea necesario. 
\clearpage
\textbf{Estimación: }2
\clearpage
\textbf{Prioridad: }C
\\
\hline
\textbf{Pruebas de aceptación:}
\begin{itemize}
    \item Modificar campos del perfil y comprobar que los cambios se guardan correctamente.
    \item Dejar campos vacíos y comprobar que no se valida el formulario.
    \item Comprobar que la información modificada se refleja al volver al perfil.
\end{itemize}
\\
\hline
\end{tabularx}
\caption{HU.4 - Modificar perfil}
\end{table}

\begin{table}[H]
\centering
\begin{tabularx}{\textwidth}{|X|}
\hline
\textbf{HU.5 - Consultar juegos utilizados anteriormente}  \\
\hline
Como usuario, necesito consultar los juegos utilizados anteriormente para revisar mi historial de uso y retomar aquellos que me resultaron útiles.
\clearpage
\textbf{Estimación: }3
\clearpage
\textbf{Prioridad: }S
\\
\hline
\textbf{Pruebas de aceptación:}
\begin{itemize}
    \item Consultar la lista de juegos utilizados previamente.
    \item Comprobar que la lista se actualiza tras jugar a un nuevo juego.
\end{itemize}
\\
\hline
\end{tabularx}
\caption{HU.5 - Consultar juegos utilizados}
\end{table}

\begin{table}[H]
\centering
\begin{tabularx}{\textwidth}{|X|}
\hline
\textbf{HU.6 - Consultar juegos no utilizados anteriormente}  \\
\hline
Como usuario, necesito consultar los juegos nuevos disponibles para poder descubrir contenidos que aún no he probado.
\clearpage
\textbf{Estimación: }3
\clearpage
\textbf{Prioridad: }M
\\
\hline
\textbf{Pruebas de aceptación:}
\begin{itemize}
    \item Acceder a la sección de juegos nuevos y visualizar los disponibles.
    \item Comprobar que desaparecen de la lista tras jugar por primera vez.
\end{itemize}
\\
\hline
\end{tabularx}
\caption{HU.6 - Consultar juegos nuevos}
\end{table}

\begin{table}[H]
\centering
\begin{tabularx}{\textwidth}{|X|}
\hline
\textbf{HU.7 - Información sobre juego}  \\
\hline
Como usuario, necesito consultar información detallada sobre un juego concreto para entender su objetivo, normas y beneficios.
\clearpage
\textbf{Estimación: }2
\clearpage
\textbf{Prioridad: }M
\\
\hline
\textbf{Pruebas de aceptación:}
\begin{itemize}
    \item Seleccionar un juego y visualizar su descripción.
    \item Comprobar que la información mostrada corresponde al juego elegido.
\end{itemize}
\\
\hline
\end{tabularx}
\caption{HU.7 - Información sobre juego}
\end{table}

\begin{table}[H]
\centering
\begin{tabularx}{\textwidth}{|X|}
\hline
\textbf{HU.8 - Ver tutorial del juego}  \\
\hline
Como usuario, necesito acceder a un tutorial de cada juego para saber cómo jugar antes de comenzar.
\clearpage
\textbf{Estimación: }7
\clearpage
\textbf{Prioridad: }S
\\
\hline
\textbf{Pruebas de aceptación:}
\begin{itemize}
    \item Acceder al tutorial correcto desde la pantalla del juego seleccionado.
    \item Verificar que el tutorial explica correctamente el funcionamiento.
\end{itemize}
\\
\hline
\end{tabularx}
\caption{HU.8 - Ver tutorial del juego}
\end{table}

\begin{table}[H]
\centering
\begin{tabularx}{\textwidth}{|X|}
\hline
\textbf{HU.9 - Jugar a \textit{Metro de la Memoria}}  \\
\hline
Como usuario, necesito jugar al minijuego \textit{Metro de la Memoria} para entrenar la memoria de trabajo de forma divertida.
\clearpage
\textbf{Estimación: }7
\clearpage
\textbf{Prioridad: }M
\\
\hline
\textbf{Pruebas de aceptación:}
\begin{itemize}
    \item Acceder al juego y completar al menos una partida.
    \item Comprobar que se registran los resultados al finalizar.
\end{itemize}
\\
\hline
\end{tabularx}
\caption{HU.9 - Jugar a \textit{Metro de la Memoria}}
\end{table}

\begin{table}[H]
\centering
\begin{tabularx}{\textwidth}{|X|}
\hline
\textbf{HU.10 - Jugar a \textit{Desván Mágico}}  \\
\hline
Como usuario, necesito jugar al minijuego \textit{Desván Mágico} para trabajar la planificación y organización.
\clearpage
\textbf{Estimación: }7
\clearpage
\textbf{Prioridad: }M
\\
\hline
\textbf{Pruebas de aceptación:}
\begin{itemize}
    \item Acceder al juego y realizar varias actividades de organización.
    \item Verificar que los resultados se registran correctamente.
\end{itemize}
\\
\hline
\end{tabularx}
\caption{HU.10 - Jugar a \textit{Desván Mágico}}
\end{table}

\begin{table}[H]
\centering
\begin{tabularx}{\textwidth}{|X|}
\hline
\textbf{HU.11 - Jugar a \textit{Detective Emociones}}  \\
\hline
Como usuario, necesito jugar a \textit{Detective Emociones} para aprender a identificar emociones en diferentes contextos.
\clearpage
\textbf{Estimación: }7
\clearpage
\textbf{Prioridad: }M
\\
\hline
\textbf{Pruebas de aceptación:}
\begin{itemize}
    \item Seleccionar emociones en diversas situaciones.
    \item Recibir retroalimentación en función de las elecciones.
\end{itemize}
\\
\hline
\end{tabularx}
\caption{HU.11 - Jugar a \textit{Detective Emociones}}
\end{table}

\begin{table}[H]
\centering
\begin{tabularx}{\textwidth}{|X|}
\hline
\textbf{HU.12 - Jugar a \textit{Semáforo Emocional}}  \\
\hline
Como usuario, necesito jugar a \textit{Semáforo Emocional} para practicar el control de impulsos mediante toma de decisiones.
\clearpage
\textbf{Estimación: }7
\clearpage
\textbf{Prioridad: }M
\\
\hline
\textbf{Pruebas de aceptación:}
\begin{itemize}
    \item Enfrentarme a diferentes estímulos y decidir cómo actuar.
    \item Evaluar la respuesta escogida y los resultados.
\end{itemize}
\\
\hline
\end{tabularx}
\caption{HU.12 - Jugar a \textit{Semáforo Emocional}}
\end{table}

\begin{table}[H]
\centering
\begin{tabularx}{\textwidth}{|X|}
\hline
\textbf{HU.13 - Ver progreso por juego}  \\
\hline
Como usuario, necesito ver mi progreso en cada juego para conocer mi evolución y motivarme a seguir practicando.
\clearpage
\textbf{Estimación: }3
\clearpage
\textbf{Prioridad: }M
\\
\hline
\textbf{Pruebas de aceptación:}
\begin{itemize}
    \item Acceder a la sección de progreso y comprobar resultados de cada juego.
    \item Observar comparativas entre sesiones.
\end{itemize}
\\
\hline
\end{tabularx}
\caption{HU.13 - Ver progreso por juego}
\end{table}

\begin{table}[H]
\centering
\begin{tabularx}{\textwidth}{|X|}
\hline
\textbf{HU.14 - Ver misiones pendientes}  \\
\hline
Como usuario, necesito consultar mis misiones pendientes para saber qué retos tengo que completar.
\clearpage
\textbf{Estimación: }3
\clearpage
\textbf{Prioridad: }C
\\
\hline
\textbf{Pruebas de aceptación:}
\begin{itemize}
    \item Acceder al listado de misiones pendientes.
    \item Acceder a la información detallada de una misión seleccionada.
    \item Verificar que el listado de misiones pendientes se actualiza tras completar una misión.
\end{itemize}
\\
\hline
\end{tabularx}
\caption{HU.14 - Ver misiones pendientes}
\end{table}

\begin{table}[H]
\centering
\begin{tabularx}{\textwidth}{|X|}
\hline
\textbf{HU.15 - Ver misiones realizadas anteriormente}  \\
\hline
Como usuario, necesito consultar las misiones realizadas anteriormente (desde que se registró la cuenta) para revisar lo que he conseguido hasta ahora.
\clearpage
\textbf{Estimación: }3
\clearpage
\textbf{Prioridad: }C
\\
\hline
\textbf{Pruebas de aceptación:}
\begin{itemize}
    \item Acceder a la sección de misiones realizadas.
    \item Acceder a la información detallada de una misión seleccionada.
    \item Comprobar que aparecen correctamente con fecha de finalización.
\end{itemize}
\\
\hline
\end{tabularx}
\caption{HU.15 - Ver misiones realizadas}
\end{table}

\begin{table}[H]
\centering
\begin{tabularx}{\textwidth}{|X|}
\hline
\textbf{HU.16 - Recibir recompensas}  \\
\hline
Como usuario, necesito recibir recompensas al realizar misiones para mantenerme motivado en el proceso de aprendizaje.
\clearpage
\textbf{Estimación: }5
\clearpage
\textbf{Prioridad: }C
\\
\hline
\textbf{Pruebas de aceptación:}
\begin{itemize}
    \item Finalizar una misión y recibir una recompensa.
    \item Verificar que las recompensas se acumulan en el perfil.
\end{itemize}
\\
\hline
\end{tabularx}
\caption{HU.16 - Recibir recompensas}
\end{table}

\begin{table}[H]
\centering
\begin{tabularx}{\textwidth}{|X|}
\hline
\textbf{HU.17 - Ver recompensas}  \\
\hline
Como usuario, necesito consultar las recompensas que ya he obtenido anteriormente (desde que se registró la cuenta) para mantener un control de estas.
\clearpage
\textbf{Estimación: }5
\clearpage
\textbf{Prioridad: }C
\\
\hline
\textbf{Pruebas de aceptación:}
\begin{itemize}
    \item Acceder a la sección de recompensas obtenidas.
    \item Comprobar que aparecen únicamente las que ya he obtenido.
\end{itemize}
\\
\hline
\end{tabularx}
\caption{HU.17 - Ver recompensas}
\end{table}

\begin{table}[H]
\centering
\begin{tabularx}{\textwidth}{|X|}
\hline
\textbf{HU.18 - Ver evolución de habilidades}  \\
\hline
Como usuario, necesito consultar la evolución (desde que se registró la cuenta) de mis habilidades cognitivas y emocionales (memoria, organización, autorregulación emocional y resolución de conflictos) para saber qué habilidades estoy mejorando.
\clearpage
\textbf{Estimación: }5
\clearpage
\textbf{Prioridad: }S
\\
\hline
\textbf{Pruebas de aceptación:}
\begin{itemize}
    \item Consultar informes gráficos.
    \item Consultar insignias y medallas obtenidas.
    \item Comparar habilidades a lo largo del tiempo.
\end{itemize}
\\
\hline
\end{tabularx}
\caption{HU.18 - Ver evolución de habilidades}
\end{table}

\begin{table}[H]
\centering
\begin{tabularx}{\textwidth}{|X|}
\hline
\textbf{HU.19 - Eliminar cuenta}  \\
\hline
Como usuario, necesito poder eliminar mi cuenta para mantener el control sobre mis datos personales.
\clearpage
\textbf{Estimación: }1
\clearpage
\textbf{Prioridad: }S
\\
\hline
\textbf{Pruebas de aceptación:}
\begin{itemize}
    \item Pulsar el botón de eliminar cuenta y comprobar que se elimina correctamente.
    \item Verificar que los datos personales se eliminan de la base de datos.
\end{itemize}
\\
\hline
\end{tabularx}
\caption{HU.19 - Eliminar cuenta}
\end{table}
\newpage

\subsubsection{Requisitos no funcionales}
Los requisitos no funcionales se refieren a los atributos de calidad de un sistema que definen su rendimiento, no sus funciones. A diferencia de los requisitos funcionales, que especifican las acciones y tareas que debe realizar un sistema, los requisitos no funcionales se centran en las características generales y el comportamiento del sistema en diversas condiciones. Abordan aspectos como el rendimiento, la usabilidad, la fiabilidad y la escalabilidad, garantizando que el sistema cumpla con los estándares de calidad y proporcione una experiencia de usuario satisfactoria \parencite{requisitos}.

A continuación, se presentan los principales requisitos no funcionales que deben cumplirse durante el desarrollo y posterior despliegue del proyecto.

\begin{table}[H]
\centering
\begin{tabularx}{\textwidth}{|X|p{0.7\textwidth}|}
\hline
\textbf{RNF.1} & Usabilidad \\
\hline
\textbf{Descripción} & La plataforma debe tener una interfaz clara, sencilla y visualmente accesible, especialmente diseñada para usuarios con dificultades en el área de la atención. \\
\hline
\end{tabularx}
\caption{RNF.1 - Usabilidad}
\end{table}

\begin{table}[H]
\centering
\begin{tabularx}{\textwidth}{|X|p{0.7\textwidth}|}
\hline
\textbf{RNF.2} & Rendimiento \\
\hline
\textbf{Descripción} & Las funcionalidades deben cargarse en menos de 2 segundos para evitar frustración en el usuario. \\
\hline
\end{tabularx}
\caption{RNF.2 - Rendimiento}
\end{table}

\begin{table}[H]
\centering
\begin{tabularx}{\textwidth}{|X|p{0.7\textwidth}|}
\hline
\textbf{RNF.3} & Compatibilidad multiplataforma \\
\hline
\textbf{Descripción} & La plataforma debe ser accesible desde dispositivos Android, iOS y navegadores modernos. \\
\hline
\end{tabularx}
\caption{RNF.3 - Compatibilidad}
\end{table}

\begin{table}[H]
\centering
\begin{tabularx}{\textwidth}{|X|p{0.7\textwidth}|}
\hline
\textbf{RNF.4} & Accesibilidad \\
\hline
\textbf{Descripción} & La interfaz debe contener elementos gráficos adaptados a personas con TDAH. \\
\hline
\end{tabularx}
\caption{RNF.4 - Accesibilidad}
\end{table}

\begin{table}[H]
\centering
\begin{tabularx}{\textwidth}{|X|p{0.7\textwidth}|}
\hline
\textbf{RNF.5} & Seguridad de datos \\
\hline
\textbf{Descripción} & La plataforma debe almacenar y transferir los datos del usuario de forma segura, cumpliendo con la normativa vigente en protección de datos, como la \textit{Ley Orgánica de Protección de Datos Personales y garantía de los derechos digitales - LPDGDD} \parencite{rgpd}. \\
\hline
\end{tabularx}
\caption{RNF.5 - Seguridad}
\end{table}

\begin{table}[H]
\centering
\begin{tabularx}{\textwidth}{|X|p{0.7\textwidth}|}
\hline
\textbf{RNF.6} & Escalabilidad \\
\hline
\textbf{Descripción} & El sistema debe permitir la incorporación de nuevos módulos o funcionalidades sin necesidad de rediseñar la arquitectura principal. \\
\hline
\end{tabularx}
\caption{RNF.6 - Escalabilidad}
\end{table}

\begin{table}[H]
\centering
\begin{tabularx}{\textwidth}{|X|p{0.7\textwidth}|}
\hline
\textbf{RNF.7} & Mantenibilidad \\
\hline
\textbf{Descripción} & El código debe estar documentado y estructurado para facilitar futuras tareas de mantenimiento y evolución del sistema. \\
\hline
\end{tabularx}
\caption{RNF.7 - Mantenibilidad}
\end{table}

\subsection{Bocetos de la plataforma}
A continuación, se muestran los bocetos realizados a mano de la plataforma, así como de cada juego de forma individual, sirviendo de base para la definición de requisitos. 
En primer lugar, se diseñó un inicio de sesión [\ref{fig:boceto_inicio}] que dirigiera a la pantalla principal de selección de juegos [\ref{fig:boceto_juegos}]. 
Esta pantalla debería mostrar los juegos, pudiendo acceder a una explicación de estos [\ref{fig:boceto_infojuego}], así como a un listado de niveles [\ref{fig:boceto_niveles}].

\begin{figure}[H]
    \centering
    \includegraphics[width=0.35\linewidth]{imgs/boceto_inicio.png}
    \caption{Boceto Inicio de sesión}
    \label{fig:boceto_inicio}
\end{figure}

\begin{figure}[H]
    \centering
    \includegraphics[width=0.35\linewidth]{imgs/boceto_juegos.png}
    \caption{Boceto Pantalla de juegos}
    \label{fig:boceto_juegos}
\end{figure}

\begin{figure}[H]
    \centering
    \includegraphics[width=0.35\linewidth]{imgs/boceto_infojuego.png}
    \caption{Boceto Información de juego}
    \label{fig:boceto_infojuego}
\end{figure}
 \begin{figure}[H]
    \centering
    \includegraphics[width=0.35\linewidth]{imgs/boceto_niveles.png}
    \caption{Boceto Pantalla de niveles}
    \label{fig:boceto_niveles}
\end{figure}

\newpage
Tras esto, se diseñaron pantallas para aportar cierta retroalimentación a los usuarios, aspecto clave para las personas con TDAH. 
De esta forma, se creó una pantalla para las estadísticas [\ref{fig:boceto_analisis}], que muestra su progreso; otra pantalla para las recompensas [\ref{fig:boceto_misiones}], en la cual se muestran las misiones a cumplir; y otra para consultar y modificar los datos del perfil del usuario [\ref{fig:boceto_perfil}].

\begin{figure}[H]
    \centering
    \includegraphics[width=0.35\linewidth]{imgs/boceto_analisis.png}
    \caption{Boceto Pantalla de estadísticas}
    \label{fig:boceto_analisis}
\end{figure}

\begin{figure}[H]
    \centering
    \includegraphics[width=0.35\linewidth]{imgs/boceto_misiones.png}
    \caption{Boceto Pantalla de misiones}
    \label{fig:boceto_misiones}
\end{figure}

\begin{figure}[H]
    \centering
    \includegraphics[width=0.35\linewidth]{imgs/boceto_perfil.png}
    \caption{Boceto Pantalla de perfil}
    \label{fig:boceto_perfil}
\end{figure}

\newpage
Respecto al diseño de los juegos, se ha optado por una estructura interactiva en la que un avatar guía al usuario a lo largo de cada actividad. 
Este avatar plantea situaciones, preguntas o retos que invitan a la reflexión o resolución de problemas, adaptados a las competencias que se desean trabajar. 
A lo largo de la experiencia, el usuario deberá resolver distintos tipos de tareas como puzles, escenarios de toma de decisiones, clasificación de elementos o identificación de emociones, en función del juego concreto. 
Esta dinámica busca mantener la atención del estudiante, fomentar su implicación activa y facilitar el aprendizaje a través del juego significativo y personalizado.

En primer lugar, en el juego \textbf{Metro de la memoria}, el avatar dará indicaciones sobre en qué parada deberá subirse y bajarse el usuario [\ref{fig:boceto_metro_1}].
Tras esto, se le proporcionará, a modo de ayuda, una frase mnemotécnica que explique el camino a tomar [\ref{fig:boceto_metro_2}].
Con toda esta información, el usuario deberá seleccionar, parada a parada, el camino a seguir para llegar al destino establecido [\ref{fig:boceto_metro_3}].

\begin{figure}[H]
    \centering
    \includegraphics[width=0.35\linewidth]{imgs/boceto_metro_1.png}
    \caption{Boceto Juego Metro de la memoria - Parte 1}
    \label{fig:boceto_metro_1}
\end{figure}

\begin{figure}[H]
    \centering
    \includegraphics[width=0.35\linewidth]{imgs/boceto_metro_2.png}
    \caption{Boceto Juego Metro de la memoria - Parte 2}
    \label{fig:boceto_metro_2}
\end{figure}

\begin{figure}[H]
    \centering
    \includegraphics[width=0.35\linewidth]{imgs/boceto_metro_3.png}
    \caption{Boceto Juego Metro de la memoria - Parte 3}
    \label{fig:boceto_metro_3}
\end{figure}

\begin{figure}[H]
    \centering
    \includegraphics[width=0.35\linewidth]{imgs/boceto_metro_4.png}
    \caption{Boceto Juego Metro de la memoria - Parte 4}
    \label{fig:boceto_metro_4}
\end{figure}

En el juego \textbf{Desván mágico}, el personaje explicará al usuario el criterio de organización al iniciar el nivel [\ref{fig:boceto_desvan_1}]. 
Tras esto, se mostrarán tanto los compartimentos como los objetos, y el usuario deberá arrastrarlos hasta la sección adecuada [\ref{fig:boceto_desvan_2}]. 
También se dará la posibilidad de consultar los objetos ya establecidos en un compartimento concreto, pudiendo eliminarlos en caso de querer añadirlos a otra sección [\ref{fig:boceto_desvan_3}]. 
Se añadirán algunos objetos a modo de "trampa", los cuales no deberán ser incorporados a ninguno de los compartimentos. 

\begin{figure}[H]
    \centering
    \includegraphics[width=0.35\linewidth]{imgs/boceto_desvan_1.png}
    \caption{Boceto Juego Desván mágico - Parte 1}
    \label{fig:boceto_desvan_1}
\end{figure}

\begin{figure}[H]
    \centering
    \includegraphics[width=0.35\linewidth]{imgs/boceto_desvan_2.png}
    \caption{Boceto Juego Desván mágico - Parte 2}
    \label{fig:boceto_desvan_2}
\end{figure}

\begin{figure}[H]
    \centering
    \includegraphics[width=0.35\linewidth]{imgs/boceto_desvan_3.png}
    \caption{Boceto Juego Desván mágico - Parte 3}
    \label{fig:boceto_desvan_3}
\end{figure}

\begin{figure}[H]
    \centering
    \includegraphics[width=0.35\linewidth]{imgs/boceto_desvan_4.png}
    \caption{Boceto Juego Desván mágico - Parte 4}
    \label{fig:boceto_desvan_4}
\end{figure}

\newpage
Respecto al juego \textbf{Detective emociones}, el avatar planteará una situación sobre un personaje ficticio [\ref{fig:boceto_detective_2}] y pedirá al usuario que analice los sentimientos que debe sentir ese personaje, mostrando una barra de nivel para una serie de sentimientos [\ref{fig:boceto_detective_4}]. 
Posteriormente, mostrará un formulario con opciones para que escoja qué habría hecho él de haber estado en esa situación [\ref{fig:boceto_detective_6}].

\begin{figure}[H]
    \centering
    \includegraphics[width=0.35\linewidth]{imgs/boceto_detective_1.png}
    \caption{Boceto Juego Detective emociones - Parte 1}
    \label{fig:boceto_detective_1}
\end{figure}

\begin{figure}[H]
    \centering
    \includegraphics[width=0.35\linewidth]{imgs/boceto_detective_2.png}
    \caption{Boceto Juego Detective emociones - Parte 2}
    \label{fig:boceto_detective_2}
\end{figure}

\begin{figure}[H]
    \centering
    \includegraphics[width=0.35\linewidth]{imgs/boceto_detective_3.png}
    \caption{Boceto Juego Detective emociones - Parte 3}
    \label{fig:boceto_detective_3}
\end{figure}

\begin{figure}[H]
    \centering
    \includegraphics[width=0.35\linewidth]{imgs/boceto_detective_4.png}
    \caption{Boceto Juego Detective emociones - Parte 4}
    \label{fig:boceto_detective_4}
\end{figure}

\begin{figure}[H]
    \centering
    \includegraphics[width=0.35\linewidth]{imgs/boceto_detective_5.png}
    \caption{Boceto Juego Detective emociones - Parte 5}
    \label{fig:boceto_detective_5}
\end{figure}

\begin{figure}[H]
    \centering
    \includegraphics[width=0.35\linewidth]{imgs/boceto_detective_6.png}
    \caption{Boceto Juego Detective emociones - Parte 6}
    \label{fig:boceto_detective_6}
\end{figure}

\begin{figure}[H]
    \centering
    \includegraphics[width=0.35\linewidth]{imgs/boceto_detective_7.png}
    \caption{Boceto Juego Detective emociones - Parte 7}
    \label{fig:boceto_detective_7}
\end{figure}

\newpage
En último lugar, se realizaron los bocetos del juego \textbf{Semáforo emocional}, en los cuales se plantea otra situación hipotética la cuál deberá analizar el usuario [\ref{fig:boceto_semaforo_1}] [\ref{fig:boceto_semaforo_2}]. 
Tras esto, se muestra una pantalla con un semáforo en rojo, la cual trata de hacer reflexionar al usuario sobre qué sentiría en esa situación [\ref{fig:boceto_semaforo_3}]; seguida de otra con un semáforo en amarillo, para tomar una decisión sobre cómo actuar [\ref{fig:boceto_semaforo_5}]; y una última pantalla con un semáforo en verde, en la que se muestra el desenlace en función de la decisión tomada [\ref{fig:boceto_semaforo_7}].

\begin{figure}[H]
    \centering
    \includegraphics[width=0.35\linewidth]{imgs/boceto_semaforo_1.png}
    \caption{Boceto Juego Semáforo emocional - Parte 1}
    \label{fig:boceto_semaforo_1}
\end{figure}

\begin{figure}[H]
    \centering
    \includegraphics[width=0.35\linewidth]{imgs/boceto_semaforo_2.png}
    \caption{Boceto Juego Semáforo emocional - Parte 2}
    \label{fig:boceto_semaforo_2}
\end{figure}

\begin{figure}[H]
    \centering
    \includegraphics[width=0.35\linewidth]{imgs/boceto_semaforo_3.png}
    \caption{Boceto Juego Semáforo emocional - Parte 3}
    \label{fig:boceto_semaforo_3}
\end{figure}

\begin{figure}[H]
    \centering
    \includegraphics[width=0.35\linewidth]{imgs/boceto_semaforo_4.png}
    \caption{Boceto Juego Semáforo emocional - Parte 4}
    \label{fig:boceto_semaforo_4}
\end{figure}

\begin{figure}[H]
    \centering
    \includegraphics[width=0.35\linewidth]{imgs/boceto_semaforo_5.png}
    \caption{Boceto Juego Semáforo emocional - Parte 5}
    \label{fig:boceto_semaforo_5}
\end{figure}

\begin{figure}[H]
    \centering
    \includegraphics[width=0.35\linewidth]{imgs/boceto_semaforo_6.png}
    \caption{Boceto Juego Semáforo emocional - Parte 6}
    \label{fig:boceto_semaforo_6}
\end{figure}

\begin{figure}[H]
    \centering
    \includegraphics[width=0.35\linewidth]{imgs/boceto_semaforo_7.png}
    \caption{Boceto Juego Semáforo emocional - Parte 7}
    \label{fig:boceto_semaforo_7}
\end{figure}

\newpage
Todos estos bocetos han sido mostrados a las tutoras del proyecto, quienes dieron su aprobación.
Estas actividades iniciales han permitido establecer una visión clara y coherente del producto a desarrollar, asegurando que cada juego responda a una necesidad concreta detectada en el contexto del TDAH.

\subsection{Comparativa de aplicaciones con el proyecto}
Tras el análisis de otras aplicaciones similares y el planteamiento del proyecto a desarrollar, se han añadido las características de nuestro proyecto a la comparación previamente realizada. Con esta información, se han obtenido las siguientes tablas resumen, numerando los juegos implementados de la siguiente forma:

\begin{enumerate}
    \item Metro de la memoria
    \item Desván mágico
    \item Detective emociones
    \item Semáforo emocional
\end{enumerate}

\begin{table}[H]
\centering
\begin{tabularx}{\textwidth}{|X|p{1cm}|p{1cm}|p{1cm}|p{1.3cm}|p{3cm}|}
    \hline
    \textbf{Competencia} & \textbf{RPA} & \textbf{Lum.} & \textbf{CG} & \textbf{Ludi Mind} & \textbf{Juegos implicados} \\
    \hline
    Autorregulación emocional         & x &   &   & x & 3,4 \\
    \hline
    Resolución de conflictos sociales & x &   &   & x & 4 \\
    \hline
    Identificación emocional          & x &   &   & x & 3 \\
    \hline
    Memoria de trabajo                &   & x & x & x & 1 \\
    \hline
    Organización de objetos/tareas    &   &   & x & x & 2 \\
    \hline
    Planificación / anticipación      &   & x & x & x & 2,4 \\
    \hline
    Control inhibitorio / impulsos    & x & x & x & x & 4 \\
    \hline
    Gestión del tiempo                &   & x &   & x & 2 \\
    \hline
    Motivación por refuerzo           & x & x & x & x & 1,2,3,4 \\
    \hline
\end{tabularx}
\caption{Nueva comparativa entre competencias de aplicaciones}
\end{table}

\begin{table}[H]
\centering
\begin{tabularx}{\textwidth}{|X|X|X|X|X|}
    \hline
    \textbf{Característica} & \textbf{Respira, Piensa, Actúa} & \textbf{Lumosity} & \textbf{CogniFit} & \textbf{LudiMind} \\
    \hline
    Precio & Gratuita & Gratis con limitación; suscripción premium para escoger juegos & Suscripción obligatoria & Gratuita \\
    \hline
    Sistema Operativo & iOS, Android & Web, iOS, Android & Web, iOS, Android & Web, iOS, Android \\
    \hline
    Edad recomendada & 2–5 años & $\ge$ 13 años & 6–18 años & 6–18 años \\
    \hline
    Nivel de dificultad & Fijo & Adaptativo & Adaptativo & Adaptativo \\
    \hline
    Perfil TDAH & No & No & Sí & Sí \\
    \hline
    Personalización & No & Sí & Sí & Sí \\
    \hline
    Estudios que evalúan su eficacia & No & Sí & Sí & Sí \\
    \hline
    Registro de usuarios y perfiles & No & Sí & Sí & Sí \\
    \hline
    Seguimiento del progreso & No & Sí & Sí & Sí \\
    \hline
    Refuerzos y recompensas & No & Sí & Sí & Sí \\
    \hline
\end{tabularx}
\caption{Comparativa de las aplicaciones según características}
\end{table}

Esta tabla muestra cómo este proyecto logra integrar de forma equilibrada competencias clave relacionadas con el TDAH que no están completamente cubiertas por aplicaciones existentes. 
Mientras que la primera aplicación estudiada se centra en la autorregulación emocional y, tanto Lumosity como CogniFit, en funciones cognitivas generales, la plataforma desarrollada combina memoria, organización, habilidades emocionales y sociales, logrando adaptarse mejor a las necesidades de niños con TDAH. 
De estas aplicaciones analizadas se han tomado numerosas ideas que servirán para el desarrollo de la plataforma, buscando obtener un resultado lo más completo y efectivo posible.
La aplicación Respira, Piensa, Actúa nos ha inspirado para la elaboración del juego Semáforo emocional, el cual también trabaja la autorregulación emocional. 
Por otro lado, Lumosity y CogniFit han servido de referencia para la estructura de la aplicación, añadiendo diferentes misiones y evaluaciones del rendimiento del usuario.

\subsection{Elección de herramientas de desarrollo}
El objetivo de este proyecto es desarrollar una plataforma web que pueda integrarse dentro de una web ya existente, por lo que se ha optado por priorizar tecnologías compatibles con la arquitectura web actual, las cuales permitan una integración fácil y modular. 
Además, deben ofrecer flexibilidad tanto en diseño como en lógica y tener buena escalabilidad y mantenimiento a largo plazo.

A continuación, se muestra una tabla comparativa con las principales opciones consideradas para el desarrollo de la plataforma, evaluadas en el contexto del proyecto.

\begin{table}[H]
\centering
\begin{tabularx}{\textwidth}{|p{3.5cm}|X|X|}
\hline
\textbf{Tecnología} & \textbf{Ventajas} & \textbf{Desventajas} \\
\hline
\parbox[t]{\hsize}{\textbf{Flutter + Dart} \\[0.5em] \vspace*{\fill} \parencite{flutter}} &
\begin{itemize}[leftmargin=*]
    \item Desarrollo multiplataforma desde un solo código.
    \item Alto rendimiento con compilación nativa.
    \item Hot Reload para agilizar pruebas.
    \item Interfaces atractivas y personalizables.
\end{itemize} &
\begin{itemize}[leftmargin=*]
    \item Ecosistema de librerías más limitado.
    \item Aplicaciones más pesadas.
    \item Acceso restringido a algunas APIs nativas.
\end{itemize} \\
\hline
\parbox[t]{\hsize}{\textbf{Node.js + React.js/Vue.js} \\[0.5em] \vspace*{\fill} \parencite{nodejs,react,vuejs}} &
\begin{itemize}[leftmargin=*]
    \item Alto rendimiento y escalabilidad.
    \item Ecosistema muy amplio con npm.
    \item Comunidad activa y soporte constante.
    \item \textbf{React}: componentes reutilizables, DOM virtual eficiente.
    \item \textbf{Vue}: aprendizaje sencillo y personalización flexible.
\end{itemize} &
\begin{itemize}[leftmargin=*]
    \item Calidad variable en librerías npm.
    \item \textbf{Node}: no apto para tareas intensivas en CPU.
    \item \textbf{React}: solo cubre la capa de interfaz.
    \item \textbf{Vue}: menor respaldo corporativo, menos soporte en español.
\end{itemize} \\
\hline
\parbox[t]{\hsize}{\textbf{Django + React.js/Vue.js} \\[0.5em] \vspace*{\fill} \parencite{django}} &
\begin{itemize}[leftmargin=*]
    \item Desarrollo rápido y alta productividad.
    \item Modularidad y reutilización de código.
    \item Buenas herramientas de seguridad integradas.
\end{itemize} &
\begin{itemize}[leftmargin=*]
    \item Problemas de escalabilidad en proyectos muy grandes.
    \item Requiere mantenimiento continuo y gestión de actualizaciones.
\end{itemize} \\
\hline
\end{tabularx}
\caption{Comparativa de tecnologías para el desarrollo de la plataforma.}
\end{table}

Tras analizar distintas opciones para el desarrollo de la plataforma, se ha decidido seleccionar React.js para el frontend y Node.js para el backend. 
Esta elección se debe a la posibilidad de un desarrollo ágil y sencillo, con una fácil comunicación entre capas. 
Además, React.js aporta un sistema basado en componentes reutilizables, ideal para construir interfaces interactivas y dinámicas, mientras que Node.js proporciona un backend eficiente, escalable y con un gran ecosistema de librerías disponibles.

Para maximizar el potencial de esta arquitectura, se ha decidido complementar Node.js con Express. 
Este es un framework que simplifica la creación de API REST y la gestión de rutas, peticiones y middleware. 
Su integración con Node.js es natural y muy utilizada, ofreciendo una estructura clara y modular que favorece el mantenimiento y la escalabilidad del proyecto \parencite{express}.

\begin{figure}[H]
    \centering
    \includegraphics[width=0.5\linewidth]{imgs/node_express.png}
    \caption{Herramientas seleccionadas - Node.js + Express}
\end{figure}

Asimismo, se ha optado por incorporar Tailwind CSS como framework de estilos. Este método permite un desarrollo ágil y evita la necesidad de escribir hojas de CSS extensas, propiciando interfaces coherentes, personalizables y responsivas. Estas características lo hacen ideal para una plataforma de juegos con una experiencia visual atractiva, moderna y fácil de mantener \parencite{tailwind}.

\begin{figure}[H]
    \centering
    \includegraphics[width=0.5\linewidth]{imgs/react_tailwind.jpg}
    \caption{Herramientas seleccionadas - React.js + Tailwind CSS}
\end{figure}

Finalmente, para la base de datos se ha escogido Supabase, una plataforma BaaS (Backend as a Service) alojada en la nube que provee a los desarrolladores una amplia gama de herramientas para crear y gestionar servicios backend. Esta herramienta ofrece todos los servicios necesarios para crear una aplicación escalable y segura: gestión de base de datos, autenticación, almacenamiento de archivos, generación automática de APIs y actualizaciones en tiempo real, entre otros. En cuanto a su base de datos, utiliza PostgreSQL relacional de código abierto, conocida por ser confiable y escalable \parencite{supabase}.

\begin{figure}[H]
    \centering
    \includegraphics[width=1\linewidth]{imgs/supabase.png}
    \caption{Herramientas seleccionadas - Supabase + PostgreSQL}
\end{figure}

En conjunto, la combinación de React.js y Tailwind CSS para el frontend, Node.js con Express para el backend y Supabase para la base de datos proporciona una base tecnológica sólida, escalable y flexible, perfectamente adaptada a los requerimientos de la plataforma y a su integración en la web ya existente.

\subsection{Diagrama de arquitectura}
Tras escoger todas las tecnologías que se utilizarán para el desarrollo de la plataforma, se ha realizado un diagrama de arquitectura que muestra dicha información de manera más clara y visual.

\begin{figure}[H]
    \centering
    \includegraphics[width=1\linewidth]{imgs/diagrama_arq.png}
    \caption{Diagrama de arquitectura}
\end{figure}

\subsection{Fases del proyecto}
\subsubsection{Sprint 0}
\paragraph{Desarrollo del sprint}
Durante esta primera fase se ha llevado a cabo una investigación preliminar que incluyó el estudio de las necesidades del público objetivo, analizando aplicaciones existentes que pudieran facilitar la comprensión y la extracción de conclusiones clave. Tras esto, se realizó una propuesta inicial con cuatro juegos, definidos a partir de las competencias clave detectadas: memoria de trabajo, planificación, autorregulación emocional y resolución de conflictos sociales.

Posteriormente, se elaboraron bocetos a mano para la aplicación, así como para cada juego de forma individual, sirviendo de base para la definición de requisitos e historias de usuario, los cuales se realizarán en la siguiente iteración.

\begin{figure}[H]
    \centering
    \includegraphics[width=1\linewidth]{imgs/sprint0.PNG}
    \caption{Backlog - Sprint 0}
\end{figure}

En esta etapa inicial no encontramos historias de usuario asociadas, pues se centró en la investigación y definición del proyecto. 
Por ello, ninguna de las tareas planteadas en este sprint son de tipo "Desarrollo".

\paragraph{Retrospectiva}
El Sprint 0 se desarrolló de forma satisfactoria, cumpliendo con todos los objetivos planteados y estableciendo una base sólida para el resto del proyecto. La investigación inicial permitió comprender en profundidad las necesidades del público objetivo, y la definición de los juegos resultó coherente con las competencias clave identificadas.

Los bocetos elaborados facilitaron la visualización del producto y servirán como referencia para las próximas fases. En general, el trabajo realizado en esta iteración ha permitido iniciar el proyecto con una dirección clara y bien fundamentada.

\subsubsection{Sprint 1}
\paragraph{Desarrollo del sprint}
Durante esta fase se continuó con la redacción de apartados como la motivación o la elección de la metodología de planificación. Además, se realizó todo el apartado de análisis, incluyendo historias de usuario, requisitos, tanto funcionales como no funcionales, y diagramas.

Todo esto nos permitió terminar de explicar qué deberá hacer la plataforma y cómo será utilizada por los usuarios, lo que facilitará el posterior desarrollo del proyecto.

\begin{figure}[H]
    \centering
    \includegraphics[width=1\linewidth]{imgs/sprint1.PNG}
    \caption{Backlog - Sprint 1}
\end{figure}

En esta etapa, al igual que en la anterior, no encontramos historias de usuario asociadas, pues está también centrada en la investigación y definición del proyecto. 

\paragraph{Retrospectiva}
En términos generales, este sprint se desarrolló de manera satisfactoria, cumpliendo con los objetivos planteados y continuando con el establecimiento de la base del proyecto. 

No obstante, se identificó que el ritmo de trabajo era mayor al esperado, pudiendo haber establecido más tareas en dicha iteración. De cara a los siguientes sprints, se planificará una mejor distribución del tiempo y un incremento de puntos de historia para maximizar el avance en cada iteración.

\subsubsection{Sprint 2}
\paragraph{Desarrollo del sprint}
En esta iteración se comenzó con el desarrollo de la plataforma. Para ello, se escogió el lenguaje de programación, planteando diversos lenguajes de programación, y se creó un repositorio de trabajo al que ir incorporando gradualmente el progreso de la implementación.

Tras esto, se desarrolló la estructura básica de la página, incluyendo la configuración inicial del entorno de desarrollo, la creación de la base de datos y la implementación de funcionalidades básicas como un inicio de sesión o un registro de usuarios.
A esto se le añadió la creación de la pantalla principal, que permite consultar los distintos juegos disponibles en la plataforma, tanto los jugados anteriormente como los nuevos.

\begin{figure}[H]
    \centering
    \includegraphics[width=1\linewidth]{imgs/sprint2.PNG}
    \caption{Backlog - Sprint 2}
\end{figure}

Respecto a las historias de usuario, en esta fase se han desarrollado cuatro de ellas, expuestas a continuación:

\begin{table}[H]
\centering
\begin{tabularx}{\textwidth}{|X|p{0.6\textwidth}|X|X|X|}
\hline
\textbf{Ident.} & \textbf{Título} & \textbf{Est.} & \textbf{Prio.} & \textbf{Iter.} \\
\hline
HU.1 & Como usuario necesito registrarme. & 2 & M & 2 \\
\hline
HU.2 & Como usuario necesito iniciar sesión. & 3 & M & 2 \\
\hline
HU.5 & Como usuario necesito consultar los juegos utilizados anteriormente. & 3 & S & 2 \\
\hline
HU.6 & Como usuario necesito consultar juegos nuevos. & 3 & M & 2 \\
\hline
\end{tabularx}
\caption{Listado de Historias de usuario - Sprint 2}
\end{table}

\paragraph{Retrospectiva}
El Sprint 2 se desarrolló también de manera satisfactoria, logrando avances significativos en la implementación de la plataforma. La elección del lenguaje de programación y la creación del repositorio de trabajo facilitaron el desarrollo de manera cómoda y la integración continua de los avances.

Sin embargo, al igual que en el sprint anterior, el ritmo de trabajo fue superior al previsto, terminando las tareas planteadas antes de lo estipulado. No obstante, se determinó que dicho suceso no ocurriría en futuras iteraciones, pues ya se comenzaría con la implementación de los juegos, 
los cuales contarán con diversos niveles. De este modo, el número de niveles añadidos podrá ir en función del ritmo de trabajo de cada iteración.
\subsubsection{Sprint 3}

\subsection{Estructura del proyecto}

\subsection{Presupuesto del proyecto}

\newpage
\section{Evaluación con usuarios}
Para evaluar la efectividad y usabilidad de los minijuegos desarrollados, se llevaron a cabo diversas evaluaciones con usuarios reales.
Estas evaluaciones se centraron en medir el grado de aceptabilidad y usabilidad de los juegos, así como en recoger feedback para futuras mejoras.
Para ello, se realizaron pruebas tanto con un grupo no clínico de estudiantes de bachillerato (grupo sin TDAH) como con un grupo clínico de niños y niñas diagnosticados con TDAH.
Como hipótesis inicial, se planteó que los juegos serían bien aceptados por ambos grupos, pudiendo utilizarlos de manera sencilla e intuitiva.

En estas pruebas se les pidió a los participantes que jugaran a un nivel de cada uno de los minijuegos desarrollados.
Durante la sesión, se recogieron datos sobre el uso de los juegos: el tiempo empleado para cada nivel, los errores cometidos, las dudas planteadas, las estrategias utilizadas por los jugadores y jugadoras y, finalmente, los resultados obtenidos.

De cara a la retroalimentación, se planteó el desarrollo tanto de un cuestionario SUS (System Usability Scale), como de un cuestionario TAM (Technology Acceptance Model), ambos adaptados al contexto de los juegos serios.
Por un lado, el cuestionario SUS se suministra a los usuarios que utilizan la aplicación (con una experiencia directa), idealmente inmediatamente después de utilizarla para que puedan evaluar su experiencia de uso de manera inmediata y precisa \parencite{sus}.
Cada pregunta se puntúa en una escala likert del 1 al 5, siendo 1 \textit{'totalmente en desacuerdo'} y 5 \textit{'totalmente de acuerdo'}, alternando preguntas positivas con negativas.
Por otro lado, el cuestionario TAM facilita la comprensión y evaluación de la aceptación de los usuarios hacia las nuevas tecnologías, permitiendo desarrollar e implementar mejores sistemas.
Se ha utilizado en muchas investigaciones, ante diversos contextos y ha demostrado ser una herramienta confiable para conocer la aceptación de estas tecnologías \parencite{tam}.

Inicialmente, se planteó el uso de ambos, pero tras la revisión de ambos con detenimiento, se observó que muchas de las preguntas del cuestionario SUS se solapaban con las del cuestionario TAM, llegando a ser redundantes.
A esto se le sumó la observación de la corta duración de las sesiones, por lo que se optó por realizar un único cuestionario TAM, añadiendo preguntas de los cuestionarios SUS \cref{fig:tam}.
El cuestionario incluyó las siguientes dimensiones:

\newpage
\begin{itemize}
    \item \textbf{F} - Facilidad de uso
    \item \textbf{DP} - Disfrute percibido
    \item \textbf{IU} - Intención de uso
    \item \textbf{C} - Compatibilidad con la rutina
    \item \textbf{AE} - Autoeficacia percibida
    \item \textbf{AC} - Aceptación global
\end{itemize}

Cada ítem se respondió mediante una escala Likert de 1 a 5, acompañada de iconos emocionales para mejorar la comprensión según la edad del usuario. 
Además, se incluyeron dos preguntas abiertas sobre errores detectados y sugerencias de mejora.
Este cuestionario puede consultarse al completo en el capítulo Anexos.

\begin{figure}[H]
    \centering
    \includegraphics[width=1\linewidth]{imgs/tam.png}
    \caption{Cuestionario TAM}
    \label{fig:tam}
\end{figure}

\newpage
\subsection{Evaluación con grupo no clínico}
La evaluación con el grupo no clínico tenía como objetivo medir la aceptabilidad, facilidad de uso e intención de uso en población escolar general.
Además, se buscaba la detección de posibles errores o dificultades en la interacción con los minijuegos, recogiendo posibles sugerencias de mejoras.
Finalmente, se trató de evaluar también la compatibilidad con la rutina con la rutina escolar de estos estudiantes.

Durante la evaluación, se contó con la participación de únicamente 6 estudiantes de bachillerato, todos ellos con 16 años.
Estas pruebas se llevaron a cabo en un entorno controlado, en una sala de la Facultad de Ciencias de la Salud de la Universidad de Granada.
A pesar del tamaño de la muestra, se pudo obtener información valiosa sobre la usabilidad y aceptación de los minijuegos.

\subsubsection{Resultados en los minijuegos}
Tras analizar los datos recogidos durante las sesiones de juego, en general, se observó que los participantes completaron los niveles de los minijuegos con éxito, mostrando un buen nivel de comprensión de las mecánicas de juego.
No obstante, se identificaron algunas áreas de mejora, las cuales se detallarán posteriormente.

\paragraph{Puntos obtenidos en los minijuegos}
Respecto a los puntos obtenidos en cada minijuego, se observó que los participantes lograron puntajes altos en la mayoría de los niveles, lo que indica que, generalmente, las mecánicas de juego eran accesibles y comprensibles.
Sin embargo, podemos observar que los resultados obtenidos en el juego \textit{Detective emociones} fueron notablemente más bajos en comparación con los otros tres minijuegos.
Esto podría deberse a que, para una situación de juego única, los participantes pueden manifestar emociones distintas, lo que aumentó la complejidad del juego en comparación con los otros minijuegos, donde las tareas eran más directas y específicas.

En definitiva, se obtuvo una media de puntuación de 2.25 sobre 3 en los minijuegos, lo que indica un buen nivel de desempeño general por parte de los participantes.

\begin{figure}[H]
    \centering
    \includegraphics[width=1\linewidth]{imgs/ncg_points_table.png}
    \caption{Grupo no clínico - Puntuaciones al jugar los minijuegos - Tabla}
    \label{fig:ncg_points_table}
\end{figure}

\begin{figure}[H]
    \centering
    \includegraphics[width=1\linewidth]{imgs/ncg_points_graphic.png}
    \caption{Grupo no clínico - Puntuaciones al jugar los minijuegos - Gráfico}
    \label{fig:ncg_points_graph}
\end{figure}

\newpage
Entre los errores cometidos, se observó que la mayoría de los participantes cometieron pocos errores en los niveles de los minijuegos.
No obstante, se pudieron identificar ciertos problemas específicos en algunos minijuegos:
\begin{itemize}
    \item En \textit{Metro de la memoria}, 3 de los 6 participantes tuvieron dificultades debido a que no recordaban correctamente la secuencia o, simplemente, no la habían leído bien.
    Esto sugiere que podría ser beneficioso incluir una opción para repetir la secuencia o proporcionar pistas adicionales.
    De esta forma, se facilitaría la experiencia de juego y se reduciría la frustración de los jugadores.
    \item En \textit{Desván mágico}, un participante falló al organizar los objetos correctamente, ya que trató de almacenarlos todos en algún cajón.
    Esto indica que, posiblemente, no leyó bien las instrucciones o no comprendió completamente la mecánica del juego.
    \item En \textit{Detective emociones}, ninguno de los participantes obtuvo la puntuación máxima.
    Esto sugiere la necesidad de plantear la posibilidad de incluir varias respuestas correctas para cada situación, ya que las emociones pueden interpretarse de diferentes maneras.
    \item En \textit{Semáforo emocional}, no se observaron errores significativos, ya que todos los participantes, a excepción de uno, lograron completar el nivel con éxito.
    Esta excepción podría atribuirse a un error puntual de atención o lectura por parte del participante.    
\end{itemize}

\paragraph{Tiempo empleado en los minijuegos}
Respecto al tiempo empleado en completar cada nivel, se observó que los participantes tardaron tiempos muy parecidos en todos los minijuegos (entre 30 y 60 segundos aproximadamente) salvo, una vez más, en \textit{Detective emociones}, donde el tiempo medio fue considerablemente mayor (llegando a alcanzar los 3 minutos y medio).
Esto se debió a que los participantes tuvieron que reflexionar más sobre las respuestas correctas, ya que las emociones pueden ser subjetivas y variar según la interpretación individual.

\begin{figure}[H]
    \centering
    \includegraphics[width=1\linewidth]{imgs/ncg_time_table.png}
    \caption{Grupo no clínico - Tiempo empleado en los minijuegos - Tabla}
    \label{fig:ncg_time_table}
\end{figure}

\begin{figure}[H]
    \centering
    \includegraphics[width=1\linewidth]{imgs/ncg_time_graphic.png}
    \caption{Grupo no clínico - Tiempo empleado en los minijuegos - Gráfico}
    \label{fig:ncg_time_graphic}
\end{figure}

\newpage
\paragraph{Áreas de mejora identificadas}
Los resultados obtenidos en los minijuegos por parte del grupo no clínico indican que, en general, los participantes pudieron comprender y completar las tareas propuestas.
No obstante, se identificaron áreas de mejora específicas en algunos minijuegos, lo que sugiere la necesidad de realizar ajustes para optimizar la experiencia de juego y facilitar la comprensión de las mecánicas.
Podemos destacar los siguientes cambios:

\begin{itemize}
    \item Incorporación de un botón de ayuda en el juego \textit{Metro de la memoria}.
    \item Incorporación de mapas más complejos en los niveles avanzados del juego \textit{Metro de la memoria}.
    \item Incorporación de más cajones y objetos en el juego \textit{Desván mágico}.
    \item Inclusión de una batería de objetos para cada nivel, mostrando diferentes objetos en cada intento en el juego \textit{Desván mágico}.
    \item Posibilidad de varias respuestas correctas en el juego \textit{Detective emociones}.
    \item Inclusión de una batería de situaciones para cada nivel tanto en el juego \textit{Detective emociones} como en el juego \textit{Semáforo emocional}, mostrando diferentes situaciones en cada intento.
    \item Almacenamiento del tiempo de juego e incorporación de un alarma si se supera un límite establecido (30 minutos aproximadamente).
    \item Posibilidad de elección del tipo de contraseña (texto o imágenes) tanto en el registro como en el inicio de sesión.
    \item Posibilidad de asignación de distractores por parte de los tutores en los juegos.
\end{itemize}

Gran parte de estas mejoras serán implementadas en futuras versiones de los minijuegos, con el objetivo de optimizar la experiencia del usuario y aumentar la efectividad de los juegos como herramientas de apoyo para niños y niñas con TDAH.

\subsubsection{Resultados del cuestionario TAM}
Tras analizar las respuestas del cuestionario TAM, las cuales fueron 4 de los 6 estudiantes que realizaron las pruebas, se obtuvieron los siguientes resultados en cada una de las dimensiones evaluadas:

\begin{itemize}
    \item \textbf{Facilidad de uso (F)}: Los participantes consideraron que los minijuegos eran fáciles de usar, con una puntuación media de 4.06 sobre 5.
    \item \textbf{Disfrute percibido (DP)}: Los estudiantes expresaron un alto nivel de disfrute al jugar, con una puntuación media de 4.25 sobre 5.
    \item \textbf{Intención de uso (IU)}: La intención de utilizar los minijuegos en el futuro fue bastante más baja que los anteriores, con una puntuación media de 2.63 sobre 5.
    \item \textbf{Compatibilidad con la rutina (C)}: Los participantes consideraron que los minijuegos eran realtivamente compatibles con su rutina diaria, obteniendo una puntuación media de 3.38 sobre 5.
    \item \textbf{Autoeficacia percibida (AE)}: Los estudiantes se sintieron bastante seguros de su capacidad para utilizar los minijuegos, con una puntuación media de 3.67 sobre 5.
    \item \textbf{Aceptación global (AC)}: La aceptación general de los minijuegos fue alta, con una puntuación media de 3.88 sobre 5.
\end{itemize}

Por otro lado, en las preguntas abiertas, los participantes indicaron que no habían encontrado ningún error significativo durante su experiencia de juego.
Sin embargo, uno de los estudiantes sugirió como mejora el aumento de la dificultad en algunos niveles, para mantener el interés y el desafío a medida que avanzaban en el juego.

En resumen, los resultados del cuestionario TAM indican que los minijuegos fueron bien recibidos por el grupo no clínico, con altas puntuaciones en facilidad de uso, disfrute percibido y aceptación global.
No obstante, la intención de uso fue relativamente baja, lo que este público podría no estar tan interesado en utilizar estos juegos en su rutina diaria.
Estos resultados proporcionan información valiosa para futuras mejoras y ajustes en los minijuegos, con el fin de optimizar la experiencia del usuario y aumentar la intención de uso.

\subsection{Evaluación con grupo clínico}
La evaluación con el grupo clínico también tenía como objetivo medir la aceptabilidad, facilidad de uso e intención de uso, pero, esta vez, en población con TDAH.
También se buscaba la detección de posibles errores o dificultades en la interacción con los minijuegos, recogiendo más sugerencias de mejoras.

En esta evaluación, se contó con 3 participantes diagnosticados con TDAH, dos de ellos con 12 años y otro con 17 años.
Esta sesión de pruebas también se llevó a cabo en un entorno controlado, en otra sala de la Facultad de Ciencias de la Salud de la Universidad de Granada.

\subsubsection{Resultados en los minijuegos}
Tras analizar los datos recogidos durante las sesiones de juego con el grupo clínico, se observó que los participantes lograron completar los niveles de los minijuegos con menor dificultad en comparación con el grupo no clínico.
Esto podría deberse a que los minijuegos fueron mejorados entre una evaluación y otra, incorporando las sugerencias obtenidas en la primera ronda de pruebas.

\paragraph{Puntos obtenidos en los minijuegos}
Respecto a los puntos obtenidos en cada minijuego, se observó que los participantes lograron puntajes altos en todos los niveles (más de 2 puntos sobre 3), a excepción del último estudiante con uno de los juegos, lo que indica que las mecánicas de juego eran accesibles y comprensibles para ellos.
Esta excepción podría atribuirse a factores individuales, como la concentración o el nivel de familiaridad con los videojuegos.

Como resumen, se obtuvo una media de puntuación de 2.41 sobre 3 en los minijuegos, lo que indica un buen nivel de desempeño general por parte de los participantes del grupo clínico.

\begin{figure}[H]
    \centering
    \includegraphics[width=1\linewidth]{imgs/cg_points_table.png}
    \caption{Grupo clínico - Puntuaciones al jugar los minijuegos - Tabla}
    \label{fig:cg_points_table}
\end{figure}

\begin{figure}[H]
    \centering
    \includegraphics[width=1\linewidth]{imgs/cg_points_graphic.png}
    \caption{Grupo clínico - Puntuaciones al jugar los minijuegos - Gráfico}
    \label{fig:cg_points_graphic}
\end{figure}

Respecto a los errores cometidos, se observó que los participantes no cometían errores en los niveles de los minijuegos, salvo algún problema puntual en la atención o lectura, pero sin llegar a afectar significativamente su desempeño en los juegos.

\paragraph{Tiempo empleado en los minijuegos}
En cuanto al tiempo empleado en completar cada nivel, se observó que los participantes tardaron tiempos semejantes a los del grupo no clínico en todos los minijuegos, obteniendo una media de 1 minuto y 13 segundos aproximadamente por nivel.

\begin{figure}[H]
    \centering
    \includegraphics[width=1\linewidth]{imgs/cg_time_table.png}
    \caption{Grupo clínico - Tiempo empleado en los minijuegos - Tabla}
    \label{fig:cg_time_table}
\end{figure}

\begin{figure}[H]
    \centering
    \includegraphics[width=1\linewidth]{imgs/cg_time_graphic.png}
    \caption{Grupo clínico - Tiempo empleado en los minijuegos - Gráfico}
    \label{fig:cg_time_graphic}
\end{figure}

\paragraph{Áreas de mejora identificadas}
Se obtuvieron resultados positivos en los minijuegos por parte del grupo clínico, indicando que los participantes pudieron comprender y completar las tareas propuestas.
En este caso, no se identificaron áreas de mejora específicas en los minijuegos, lo que sugiere que las mejoras implementadas tras la evaluación con el grupo no clínico fueron efectivas.

\subsubsection{Resultados del cuestionario TAM}
Tras analizar las respuestas del cuestionario TAM, las cuales fueron 2 de los 3 estudiantes que realizaron las pruebas, se obtuvieron los siguientes resultados en cada una de las dimensiones:

\begin{itemize}
    \item \textbf{Facilidad de uso (F)}: Los participantes de este grupo encontraron los minijuegos más fáciles de usar que el grupo no clínico, con una puntuación media de 4.39 sobre 5, lo que indica una mejora en esta dimensión.
    \item \textbf{Disfrute percibido (DP)}: Los estudiantes expresaron un bajo nivel de disfrute al jugar, con una puntuación media de 2.25 sobre 5.
    \item \textbf{Intención de uso (IU)}: La intención de utilizar los minijuegos en el futuro fue también bastante baja, con una puntuación media de 2.75 sobre 5.
    \item \textbf{Compatibilidad con la rutina (C)}: Los participantes consideraron que los minijuegos eran relativamente compatibles con su rutina diaria, obteniendo una puntuación media de 3.00 sobre 5.
    \item \textbf{Autoeficacia percibida (AE)}: Los estudiantes se sintieron muy seguros de su capacidad para utilizar los minijuegos, con una puntuación media de 4.17 sobre 5.
    \item \textbf{Aceptación global (AC)}: La aceptación general de los minijuegos fue también relativamente alta, con una puntuación media de 3.75 sobre 5.
\end{itemize}

Por otro lado, respecto a las preguntas abiertas, los participantes no indicaron de haber encontrado errores significativos durante su experiencia de juego.
Sin embargo, uno de los estudiantes sugirió como mejora la incorporación de niveles que no se realizaran de manera tan rápida, lo cual podría ayudar a entrenar su atención durante el juego.

En resumen, se observa que los minijuegos pueden resultar de ayuda para los niños y niñas con TDAH, ya que lograron comprender y completar las tareas propuestas.
No obstante, el bajo nivel de disfrute percibido y la baja intención de uso sugieren que es posible que estos juegos no sean lo suficientemente atractivos para este grupo.
Esto podría resolverse con la incorporación de la gamifificación, la cual estaba prevista para futuras versiones de los minijuegos.

\newpage
\section{Anexos}
\subsection{Cuestionario TAM de evaluación de LudiMind}
\subsubsection{Introducción}
Este cuestionario tiene como objetivo conocer tu opinión sobre el uso de LudiMind. No hay respuestas correctas o incorrectas: lo importante es lo que tú piensas y sientes.
Tus respuestas serán confidenciales y se utilizarán únicamente para mejorar la experiencia de uso y adaptar mejor las herramientas a tus necesidades. 
Por favor, responde con sinceridad a cada afirmación seleccionando la opción que mejor refleje tu grado de acuerdo. Usamos una escala de 5 puntos:

\begin{enumerate}
    \item Totalmente en desacuerdo
    \item En desacuerdo
    \item Ni de acuerdo ni en desacuerdo
    \item De acuerdo
    \item Totalmente de acuerdo
\end{enumerate}

\subsubsection{Preguntas de carácter general}

\begin{itemize}
    \item ¿Cuál es tu edad? 
    \item ¿Cuál es tu genero? 
    Hombre, mujer, prefiero no decirlo 
    \item ¿Cuál es tu rol en el proceso? 
    Estudiante, profesor, padre/madre, terapeuta
\end{itemize}

\subsubsection{Preguntas del cuestionario TAM}
\begin{table}[H]
\centering
\begin{tabularx}{\textwidth}{|c|X|}
\hline
\textbf{Código} & \textbf{Afirmación} \\
\hline
F1 & La forma de usar LudiMind es clara y comprensible \\
\hline
F2 & La forma de usar LudiMind es intuitiva \\
\hline
F3 & El uso de LudiMind no requiere mucho esfuerzo mental ni conocimientos tecnológicos \\
\hline
F4 & Pienso que es fácil de usar LudiMind \\
\hline
F5 & Es fácil aprender a usar LudiMind \\
\hline
F6 & La forma de uso es fácil y comprensible después de la fase de aprendizaje \\
\hline
F7 & Pienso que LudiMind se puede usar sin ayuda técnica \\
\hline
F8 & Valoro positivamente las ayudas y refuerzos auditivos y visuales de LudiMind \\
\hline
F9 & Cuando utilizo LudiMind, sé qué estoy haciendo en cada momento \\
\hline
DP1 & Es divertido o agradable usar las herramientas y juegos disponibles \\
\hline
DP2 & LudiMind incluye actividades adaptadas a los intereses de cada persona \\
\hline
IU1 & Tengo intención de usar LudiMind \\
\hline
IU2 & Uso juegos similares a LudiMind con frecuencia \\
\hline
C1 & LudiMind es compatible con el trabajo en el centro educativo o en casa \\
\hline
C2 & El uso de LudiMind encaja en la rutina diaria \\
\hline
AE1 & Tengo confianza en que puedo aprender a usar LudiMind fácilmente \\
\hline
AE2 & Me siento seguro/a usando LudiMind \\
\hline
AE3 & Tengo las habilidades necesarias para utilizar LudiMind \\
\hline
AC1 & Utilizar LudiMind para enseñar habilidades de organización y planificación es una buena idea \\
\hline
AC2 & Valoro positivamente herramientas como LudiMind \\
\hline
\end{tabularx}
\caption{Cuestionario TAM de evaluación de LudiMind}
\label{tab:tam_ludimind}
\end{table}

\subsubsection{Preguntas abiertas}
\begin{itemize}
    \item ¿Ha encontrado errores al utilizar los juegos? En caso afirmativo, indique cuáles.
    \item ¿Qué mejoras haría en los juegos?
\end{itemize}

\subsubsection{Consentimiento}
\begin{itemize}
    \item He leído la información proporcionada y doy mi consentimiento informado para participar en este estudio.
    \item Estoy de acuerdo en participar de manera voluntaria en la investigación.
    \item Autorizo el uso de mis datos de forma anónima y confidencial con fines exclusivamente de investigación.
\end{itemize}


% \newpage
% \input{secciones/4_mecanicajuegos}

% Bibliografía
\newpage
\section{Bibliografía}
\defbibheading{empty}{}
\renewcommand{\bibitemsep}{1em}
\printbibliography[heading=empty]

\end{document}