\section{Introducción}

\subsection{Motivación}
El TDAH, también conocido como Trastorno por Déficit de Atención e Hiperactividad, es una afección crónica que experimentan millones de niños y niñas, continuando en muchos casos en la edad adulta. La prevalencia mundial en la etapa infantil se encuentra entre el 5 y el 7, siendo el trastorno que produce el mayor número de consultas en pediatría \parencite{llanos2019tdah}. Entre los principales síntomas que manifiestan este colectivo encontramos la falta de atención y una conducta hiperactiva e impulsiva \parencite{leffa2022adhd}. 

A pesar de la existencia de recursos educativos generales, pocos están específicamente diseñados para adaptarse a las características cognitivas y emocionales del TDAH. En este contexto, una aplicación basada en juegos interactivos puede convertirse en una herramienta eficaz y motivadora, promoviendo el aprendizaje a través del juego y favoreciendo el desarrollo de las áreas clave afectadas por el TDAH.

Por otro lado, a pesar del gran número de personas que lo experimentan, la población actual no está totalmente concienciada del problema que esto supone para este colectivo. De esta forma, escogí este proyecto no solo en busca de ayudar a estas personas a organizarse y llevar a cabo actividades más fácilmente, sino que también tratando de buscar una mayor conciencia y comprensión de la población sobre el TDAH, promoviendo un enfoque más inclusivo hacia las personas que viven esta condición diariamente.

Además, este proyecto resulta muy significativo para mí al tener un familiar cercano con TDAH, lo que me ha permitido también observar personalmente las dificultades cotidianas a las que se enfrentan dichas personas, tanto en el ámbito escolar como en su vida diaria. Esto se vio reforzado por mi experiencia previa trabajando con niños y  niñas con TDAH durante el desarrollo de mi Trabajo de Fin de Grado, donde pude profundizar en sus necesidades al desarrollar otra plataforma para este público. Así pues, con este proyecto aspiro a afianzar mis conocimientos en diseño centrado en el usuario y la accesibilidad, al tiempo que contribuyo a la inclusión y comprensión del TDAH en la sociedad.

\newpage

\subsection{Objetivos del proyecto}
Este proyecto se plantea como una herramienta lúdica y educativa destinada a estudiantes con TDAH, con el fin de apoyar el desarrollo de sus habilidades cognitivas, tales como la atención, la memoria de trabajo y la supervisión de la acción.

Entre los objetivos de este proyecto destacan:
\begin{enumerate}
    \item Revisar aplicaciones similares a la planteada para la extracción de características positivas y negativas.
    \item Revisar tecnologías potenciales a ser utilizadas para poder comparar y decidir sobre las más idónea.
    \item Revisar trabajos de investigación relacionados
    \item Desarrollar una plataforma con interfaz accesible, atractiva y adaptada a la población con TDAH.
    \item Diseñar cuatro minijuegos con distintas mecánicas que trabajen áreas concretas.
    \item Incorporar elementos de gamificación que motiven a los usuarios mediante misiones y recompensas.
    \item Implementar un apartado de retroalimentación que permita al usuario consultar su evolución de forma visual y sencilla.
    \item Evaluar la usabilidad de la herramienta mediante pruebas de uso con perfiles representativos del público objetivo, ya sean estudiantes, familias o profesores.
\end{enumerate}

Con ello, se espera ofrecer un recurso innovador y totalmente funcional que se adapte a las necesidades de este público y contribuya de forma lúdica al desarrollo de los niños y niñas con TDAH.
\newpage

\subsection{Estructura del documento}

En este apartado se describe la organización del presente documento, detallando de forma resumida el contenido que se abordará en cada sección. La finalidad de esta estructura es ofrecer al lector una visión clara y ordenada del desarrollo del trabajo, facilitando la localización de la información y la comprensión.

La memoria se compone de las siguientes secciones principales:

\begin{itemize}
    \item \textbf{Introducción}: presenta la motivación del proyecto, sus objetivos y la estructura general del documento.
    \item \textbf{Estado del arte}: expone artículos científicos previos y el análisis de soluciones similares, así como la identificación de las necesidades del usuario.
    \item \textbf{Análisis}: recoge las historias de usuario, la especificación de requisitos y la representación gráfica mediante diagramas de casos de uso.
    \item \textbf{Propuesta}: define la solución planteada, las herramientas de desarrollo seleccionadas, la planificación de las fases del proyecto y su estructura.
    \item \textbf{Mecánica de los juegos}: describe en detalle el funcionamiento de cada uno de los juegos que integran la plataforma.
    \item \textbf{Bibliografía}: relación de todas las fuentes consultadas para la elaboración de la memoria.
\end{itemize}
