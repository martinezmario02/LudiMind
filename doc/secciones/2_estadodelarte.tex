\section{Estado del arte}

\subsection{Artículos científicos relacionados}
Para el diseño de una aplicación dirigida a niños y niñas con TDAH, resulta esencial revisar la evidencia científica disponible sobre la eficacia de las herramientas tecnológicas en este ámbito. Así pues, se destacan dos artículos recientes que ofrecen claros resultados positivos ante las intervenciones tecnológicas sobre distintas funciones cognitivas y conductuales afectadas por el TDAH.

En primer lugar, encontramos el artículo \textit{Meta-analysis of the efficacy of digital therapies in children with attention-deficit hyperactivity disorder} \parencite{tech_interventions_tdah}, cuyo objetivo es la evaluación de la efectividad de intervenciones digitales para mejorar los síntomas del TDAH en estudiantes. Este estudio sigue las directrices PRISMA y el Manual Cochrane \parencite{prisma}, lo cual indica que la revisión es sistemática y fiable; además de estar registrado en PROSPERO \parencite{prospero}, lo que añade también transparencia al proceso.

Para la evaluación se utilizan instrumentos que podrían ser útiles para la validación de la plataforma. Por un lado, se utiliza la \textit{Escala de Calificación del Trastorno por Déficit de Atención e Hiperactividad (ADHD-RS)} para medir la inatención e impulsividad del alumnado \parencite{adhdrs}. Para ello, se basa en los criterios del \textit{Manual Diagnóstico y Estadístico de los Trastornos Mentales, Cuarta Edición (DSM-IV)}, el cual contiene descripciones detalladas de los diferentes trastornos mentales, incluyendo criterios diagnósticos específicos para cada uno \parencite{dsmiv}. Por otro lado, también se utiliza la herramienta \textit{Evaluación Conductual de la Función Ejecutiva-2 (BRIEF-2)}, un cuestionario completado por padres o tutores que mide las funciones ejecutivas, evaluando los aspectos más cotidianos y conductuales \parencite{brief2}.

En segundo lugar, encontramos el artículo \textit{Effectiveness of Technology-Based Interventions for School-Age Children With Attention-Deficit/Hyperactivity Disorder: Systematic Review and Meta-Analysis of Randomized Controlled Trials} \parencite{meta_digital_adhd}, cuyo objetivo es la realización de una revisión sistemática de las intervenciones tecnológicas para el alumnado con TDAH, así como un metaanálisis de los resultados tras estas intervenciones. Este estudio también está registrado en PROSPERO y sigue las directrices PRISMA para garantizar el rigor metodológico.

En este caso, se utilizaron herramientas de evaluación como las escalas de calificación, los cuestionarios o las pruebas estandarizadas relacionadas con síntomas de inatención e impulsividad. Aunque el artículo no detalla los nombres exactos de las herramientas utilizadas en cada ECA, sí especifica que las medidas empleadas evaluaban aspectos como inhibición, memoria de trabajo, atención, planificación, metacognición y calidad de vida. Esta metodología puede servir como referencia para la selección de instrumentos validados en futuras investigaciones, como por ejemplo el ADHD-RS o el BRIEF-2, ya utilizados en otros estudios similares.

\subsection{Necesidades del usuario}
El trastorno por déficit de atención e hiperactividad (TDAH) es un síndrome de falta de atención, hiperactividad e impulsividad. Hay 3 tipos de TDAH según predomine la falta de atención, la hiperactividad/impulsividad o una combinación de ambas \parencite{msd_tdah_profesional}.

Los estudiantes con TDAH presentan una gran variedad de dificultades en diversas áreas del desarrollo cognitivo, emocional y social. Esto afecta de forma considerable a su vida cotidiana y su desempeño escolar \parencite{fundacioncadah}. Entre los desafíos más frecuentes se encuentran la memoria de trabajo, la planificación, la autorregulación emocional, la resolución de conflictos sociales y la gestión del tiempo.

Todas estas dificultades desembocan en problemas como el olvido de instrucciones, la dificultad para el inicio de tareas o la reacción de forma impulsiva ante la frustración. Así pues, estos niños y niñas pueden llegar a experimentar rechazo social o conflictos por su impulsividad y dificultad para interpretar normas sociales.

Además, la normativa educativa en España destaca la gran utilidad de herramientas tecnológicas dedicadas a la atención de las necesidades específicas del alumnado. De acuerdo con la Ley Orgánica 3/2020 (LOMLOE), las administraciones educativas deben asegurar los recursos necesarios para que los estudiantes que requieran una atención educativa diferente a la ordinaria, por presentar necesidades educativas especiales (en lo que se incluye el TDAH), puedan alcanzar el máximo desarrollo posible de sus capacidades personales y, a ser posible, los objetivos establecidos con carácter general para todo el alumnado \parencite{lomloe_boe}.

Por otro lado, asociaciones como FEAADAH han destacado la importancia de herramientas tecnológicas adaptadas que fomenten una educación inclusiva y equitativa, contribuyendo a la atención de la diversidad de forma más eficaz. Este enfoque coincide con el planteamiento de la plataforma DiversiAE, de la que FEAADAH forma parte \parencite{feaadah_diversiae}.

Así pues, se ha demostrado que las personas con TDAH responden de forma positiva cuando se les ofrecen entornos estructurados, rutinas fijas y dinámicas motivadoras \parencite{tech_interventions_tdah}. Es por esto que los juegos interactivos propuestos podrían ser una herramienta eficaz para estimular habilidades cognitivas y emocionales, favoreciendo el aprendizaje mediante el juego.

\subsection{Propuestas similares}
Antes de comenzar con el desarrollo de la aplicación, se ha realizado una búsqueda de aplicaciones que pudieran servir de guía, extrayendo tanto los puntos fuertes como los débiles.

\subsubsection{Respira, piensa, actúa}
Este primer juego \parencite{rpa} trata de ayudar a niños y niñas a aprender cómo resolver problemas siguiendo el siguiente patrón: respirar, pensar y actuar. Para ello, los usuarios deben tocar la pantalla para ayudar al monstruo a respirar profundamente, pensar en un plan y, finalmente, actuar de forma adecuada. Además, incorpora algunos juegos, como la posibilidad de explotar burbujas para ayudar al monstruo.

\begin{figure}[H]
    \centering
    \includegraphics[width=0.5\linewidth]{imgs/app_rpa.png}
    \caption{Respira, piensa, actúa}
\end{figure}

Este juego fomenta la autorregulación emocional en la población de menor edad, incorporando una narrativa guiada por personajes, así como la resolución de diversas situaciones. A través de diversas escenas cotidianas, el estudiante ayuda al personaje a identificar cómo se siente, a calmarse utilizando una técnica de respiración guiada y a buscar soluciones. Un ejemplo de esto es una situación en la que el personaje no puede abrocharse el abrigo.

\begin{figure}[H]
    \centering
    \includegraphics[width=0.5\linewidth]{imgs/app_rpa_burbujas.jpg}
    \caption{Minijuego de burbujas}
\end{figure}

No obstante, la aplicación a desarrollar en este proyecto debe estar adaptada a un perfil algo más maduro (entre 6 y 18 años aproximadamente) y con necesidades específicas. Además, nuestro proyecto trata de mejorar la autonomía y el pensamiento crítico mediante la toma de decisiones y retos, mientras que este otro juego sigue una narrativa guiada (pulsar para avanzar). Aun así, la estructura narrativa y el enfoque emocional de esta aplicación pueden servir como inspiración para diseñar experiencias similares en algunos minijuegos del proyecto.

\subsubsection{Lumosity - Entrenador Cerebral}
Esta otra aplicación \parencite{lumosity} consta de una gran variedad de juegos para ejercitar la memoria y entrenar el cerebro de forma interactiva y amena. Estos juegos están divididos en función del área a entrenar, tal y como se busca con nuestro proyecto.

\begin{figure}[H]
    \centering
    \includegraphics[width=0.5\linewidth]{imgs/app_lumosity.png}
    \caption{Lumosity - Entrenador Cerebral}
\end{figure}

Entre las áreas que trabaja, se encuentra la memoria de trabajo y la planificación, aspectos clave a mejorar en el caso de las personas con TDAH, ya que están directamente relacionados con su capacidad para organizar tareas, seguir instrucciones y mantener la información relevante en mente durante la realización de una actividad. En este sentido, algunos juegos de la plataforma \textit{Lumosity} como \textit{Memory Matrix} estimulan la retención activa de patrones o secuencias visuales, reforzando la memoria de trabajo. Para ello, el usuario tiene que memorizar qué celdas de una matriz están coloreadas para, posteriormente, seleccionarlas.

\begin{figure}[H]
    \centering
    \includegraphics[width=0.5\linewidth]{imgs/app_lumosity_mm.jpg}
    \caption{Minijuego Memory Matrix}
\end{figure}

Por otro lado, juegos como Pirate Passage, que requiere planificar rutas óptimas para alcanzar un objetivo, fomentan habilidades de planificación y toma de decisiones. 

\begin{figure}[H]
    \centering
    \includegraphics[width=0.5\linewidth]{imgs/app_lumosity_ps.png}
    \caption{Minijuego Pirate Passage}
\end{figure}

Además, contiene una página con estadísticas que muestra el desempeño del usuario, lo cual puede resultar muy útil para personas con TDAH, ya que de este modo, pueden recibir retroalimentación acerca de sus logros y aspectos a mejorar. 

Sin embargo, aunque estos juegos se adaptan al nivel de los usuarios, algunos de ellos pueden llegar a resultar algo complejos cuando el usuario va avanzando, pues están realizados para un intervalo de edad mayor. Nuestra plataforma, por el contrario, está centrada en un público más joven e inexperto, por lo que los juegos que se buscan tendrán una dificultad algo menor.
\newpage

\subsubsection{Cognifit - Test y Juegos}
Esta tercera aplicación \parencite{cognifit} es una plataforma de estimulación cognitiva basada en la neuropsicología y en la evaluación del rendimiento, permitiendo tanto la estimulación como el seguimiento personalizado.

\begin{figure}[H]
    \centering
    \includegraphics[width=0.5\linewidth]{imgs/app_cognifit.jpg}
    \caption{Cognifit - Test y Juegos}
\end{figure}

Una de sus principales fortalezas es la posibilidad de crear programas adaptados a perfiles específicos como el TDAH. Además, incluye juegos interactivos diseñados para fortalecer áreas cognitivas comúnmente afectadas en este trastorno, como la memoria de trabajo, la atención sostenida o la planificación.

Esta plataforma ofrece evaluaciones iniciales y periódicas, lo que permite que los juegos se adapten dinámicamente al nivel del usuario, consiguiendo un entrenamiento personalizado y progresivo.

Por ejemplo, para la memoria de trabajo, \textit{CogniFit} incluye tareas que requieren retener y manipular información por breves períodos, así como vídeos explicativos para cada área, lo que resulta útil en intervenciones orientadas a la mejora del rendimiento académico. A continuación, se muestra un minijuego en el que la plataforma muestra un número al usuario durante un segundo y este tiene que recordarlo y completar la ecuación disparando al objeto correcto.

\begin{figure}[H]
    \centering
    \includegraphics[width=0.5\linewidth]{imgs/app_cognifit_memory.jpg}
    \caption{Minijuego de memoria}
\end{figure}

Además, la plataforma puede ser utilizada tanto en entornos clínicos como en el hogar o el contexto escolar, y permite la supervisión por parte de profesionales, educadores o familias.

En definitiva, \textit{CogniFit} representa una herramienta digital útil para trabajar habilidades cognitivas específicas de forma lúdica y motivadora, algo destacable para el TDAH, donde el componente motivacional y la retroalimentación inmediata son fundamentales.

\subsection{Comparativa de aplicaciones}
Tras el análisis de otras aplicaciones, se ha realizado un estudio de aspectos clave como las competencias trabajadas en cada una de estas aplicaciones, el precio de uso,  el sistema operativo requerido o la edad a la que está dirigida. Con esta información, se han obtenido las siguientes tablas resumen:

\begin{table}[H]
\centering
\begin{tabularx}{\textwidth}{|X|X|X|X|}
    \hline
    \textbf{Competencia} & \textbf{Respira, Piensa, Actúa} & \textbf{Lumosity} & \textbf{CogniFit}\\
    \hline
    Autorregulación emocional         & x &   &   \\
    \hline
    Resolución de conflictos sociales & x &   &   \\
    \hline
    Identificación emocional          & x &   &   \\
    \hline
    Memoria de trabajo                &   & x & x \\
    \hline
    Organización de objetos/tareas    &   &   & x \\
    \hline
    Planificación / anticipación      &   & x & x \\
    \hline
    Control inhibitorio / impulsos    & x & x & x \\
    \hline
    Gestión del tiempo                &   & x &   \\
    \hline
    Motivación por refuerzo           & x & x & x \\
    \hline
\end{tabularx}
\caption{Comparativa entre competencias de aplicaciones}
\end{table}

\begin{table}[H]
\centering
\begin{tabularx}{\textwidth}{|X|X|X|X|}
    \hline
    \textbf{Aplicación} & \textbf{Precio} & \textbf{Sist. Operativo} & \textbf{Edad recomendada} \\
    \hline
    Respira, Piensa, Actúa & Gratuita & iOS, Android & 2–5 años \\
    \hline
    Lumosity & Gratis con limitación; suscripción premium para escoger juegos & Web, iOS, Android & $\ge$ 13 años \\
    \hline
    CogniFit & Suscripción obligatoria & Web, iOS, Android & 6–18 años \\
    \hline
\end{tabularx}
\caption{Comparativa entre otras áreas de las aplicaciones}
\end{table}
