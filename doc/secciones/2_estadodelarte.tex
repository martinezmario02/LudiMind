\section{Estado del arte}

\subsection{Características del TDAH}
El trastorno por déficit de atención con hiperactividad (TDAH) es un trastorno del neurodesarrollo \parencite{brown2007}.
Se diagnostica frecuentemente en la infancia y suele persistir en la edad adulta. 
Este trastorno se caracteriza por la inatención (fallos en la regulación de la atención, distracción moderada a grave, períodos de atención breve o excesiva), 
hiperactividad (inquietud principalmente mental, no en todos los subtipos) 
y comportamiento impulsivo (inestabilidad emocional y conductas impulsivas, incluyendo la inquietud motora) que produce problemas en múltiples áreas de funcionamiento, dificultando el desarrollo social, emocional y cognitivo de la persona que lo tiene \parencite{marcelo2013}.
Una característica común, y que parece ser contraria al nombre del trastorno, es la capacidad de hiperfoco o hiperconcentración, debido a que las personas con TDAH poseen una atención ligada a factores motivacionales, no conscientes, y pueden prestar atención excesiva a aquello que les proporciona recompensa inmediata \parencite{kooij2019}.
Podemos encontrar 3 tipos de TDAH según predomine la falta de atención, la hiperactividad/impulsividad o una combinación de ambas \parencite{msd_tdah_profesional}.

Los estudiantes con TDAH presentan una gran variedad de dificultades en diversas áreas del desarrollo cognitivo, emocional y social. 
Esto afecta de forma considerable a su vida cotidiana y su desempeño escolar \parencite{fundacioncadah}. 
Entre las áreas afectadas de forma más frecuente se encuentran la memoria de trabajo, la planificación, la autorregulación emocional, la resolución de conflictos sociales y la gestión del tiempo.

Todas estas dificultades desembocan en problemas como el olvido de instrucciones, la dificultad para el inicio de tareas, o la reacción de forma impulsiva ante la frustración. 
Así pues, estos niños y niñas pueden llegar a experimentar rechazo social o conflictos por su impulsividad y dificultad para interpretar normas sociales.

Además, la normativa educativa en España destaca la gran utilidad de herramientas tecnológicas dedicadas a la atención de las necesidades específicas del alumnado. 
De acuerdo con la Ley Orgánica 3/2020 (LOMLOE), las administraciones educativas deben asegurar los recursos necesarios para que los estudiantes que requieran una atención educativa diferente a la ordinaria, por presentar necesidades educativas especiales (en lo que incluye el TDAH), para que puedan alcanzar el máximo desarrollo posible de sus capacidades personales y, a ser posible, lograr los objetivos establecidos con carácter general para todo el alumnado \parencite{lomloe_boe}.
\\

Por otro lado, asociaciones como la Federación Española de Asociaciones de TDAH (FEAADAH) han destacado la importancia de desarrollar y usar herramientas tecnológicas adaptadas que fomenten una educación inclusiva y equitativa, contribuyendo a la atención de la diversidad de forma más eficaz. 
Este enfoque coincide con el planteamiento de la plataforma DiversiAE, de la que FEAADAH forma parte \parencite{feaadah_diversiae}.

Así pues, se ha demostrado que las personas con TDAH responden de forma positiva cuando se les ofrecen entornos estructurados, rutinas fijas y dinámicas motivadoras \parencite{tech_interventions_tdah}. 
Diversos estudios recientes también destacan que el uso de juegos serios y tecnologías interactivas puede tener un impacto positivo en la mejora de la atención, la memoria de trabajo y la autorregulación en población infantil con TDAH \parencite{doulou2025}. 
Es por esto que los juegos interactivos propuestos podrían ser una herramienta eficaz para estimular habilidades cognitivas y emocionales, favoreciendo el aprendizaje mediante el juego.

De este modo, se plantea como proyecto la implementación de una plataforma de minijuegos que pueda incorporarse al proyecto \textit{DiverAccion}, el cual pretende mejorar habilidades como la memoria, planificación, organización y gestión del tiempo en adolescentes con diagnóstico de TDAH, entre 9 y 17 años. 
Este proyecto constará de un sistema de tele-rehabilitación en colaboración con los principales agentes implicados en el proceso de intervención en las actividades diarias y académicas \parencite{diveraccion}.

\subsection{Juegos serios para personas con NEAE}
Los juegos serios son aplicaciones diseñadas con un fin educativo o terapéutico que combinan elementos lúdicos con objetivos específicos de aprendizaje o mejora de habilidades.
En el contexto de las personas con Necesidades Específicas de Apoyo Educativo (NEAE), los juegos serios pueden ser una herramienta eficaz para abordar diversas áreas de desarrollo, como la cognición, la motricidad, la socialización y la autorregulación emocional.
Estos juegos permiten a los usuarios interactuar con entornos controlados donde puedan practicar habilidades sin la presión de un entorno clínico.
Esto puede facilitar el ritmo de aprendizaje y la retroalimentación inmediata \parencite{neurekalab2022}.

En el caso específico de personas con TDAH, estos juegos serios han mostrado ser muy efectivos para mejorar la atención sostenida y la memoria de trabajo.
Un estudio realizado en Cataluña implementó un programa que contaba con diversos juegos serios, incluyendo actividades como selección de estímulos o laberintos.
Los resultados obtenidos indicaron mejoras significativas en la atención sostenida y la memoria de trabajo en niños y niñas con TDAH tras la intervención \parencite{redalyc2019}.

También se han desarrollado juegos destinados a la evaluación de características de este trastorno, investigando para el desarrollo de herramientas que ayudaran en el diagnóstico del TDAH mediante la evaluación de funciones ejecutivas \parencite{pmc2014};
así como la mejora de la adherencia al tratamiento al hacer las intervenciones más atractivas y motivadoras.

Así pues, es crucial que los juegos serios sean diseñados incorporando recompensas y elementos de gamificación que fomenten la participación activa de los usuarios; así como niveles progresivos y retroalimentación positiva para mantener el interés. Además, se deben diseñar actividades que trabajen áreas clave afectadas por el TDAH, como la memoria, la organización o la regulación emocional.

\subsection{Propuestas similares}
Antes de comenzar con el desarrollo de la aplicación, se ha realizado una búsqueda de aplicaciones que pudieran servir de guía, extrayendo tanto los puntos fuertes como los débiles.

\subsubsection{Respira, piensa, actúa}
Este primer juego \parencite{rpa} trata de ayudar a niños y niñas a aprender cómo resolver problemas siguiendo el siguiente patrón: respirar, pensar y actuar. 
Para ello, los usuarios deben tocar la pantalla para ayudar al monstruo a respirar profundamente, pensar en un plan y, finalmente, actuar de forma adecuada. 
Además, incorpora algunos juegos, como la posibilidad de explotar burbujas para ayudar al monstruo.
Esto permite al usuario practicar la respiración profunda, una técnica de autorregulación emocional que puede ser beneficiosa para personas con TDAH.

\begin{figure}[H]
    \centering
    \includegraphics[width=0.5\linewidth]{imgs/app_rpa.png}
    \caption{Respira, piensa, actúa}
\end{figure}

Este juego fomenta la autorregulación emocional en la población de menor edad, incorporando una narrativa guiada por personajes, así como la resolución de diversas situaciones. 
A través de diversas escenas cotidianas, el estudiante ayuda al personaje a identificar cómo se siente, a calmarse utilizando una técnica de respiración guiada y a buscar soluciones. 
Un ejemplo de esto es una situación en la que el personaje no puede abrocharse el abrigo.

\begin{figure}[H]
    \centering
    \includegraphics[width=0.5\linewidth]{imgs/app_rpa_burbujas.jpg}
    \caption{Minijuego de burbujas}
\end{figure}

No obstante, la aplicación a desarrollar en este proyecto debe estar adaptada a un perfil algo más maduro (entre 9 y 17 años aproximadamente) y con necesidades específicas. 
Además, nuestro proyecto trata de mejorar la autonomía y el pensamiento crítico mediante la toma de decisiones y retos, mientras que este otro juego sigue una narrativa guiada (pulsar para avanzar). 
Aun así, la estructura narrativa y el enfoque emocional de esta aplicación pueden servir como inspiración para diseñar experiencias similares en algunos minijuegos del proyecto.

\subsubsection{Lumosity - Entrenador Cerebral}
Esta otra aplicación \parencite{lumosity} consta de una gran variedad de juegos para ejercitar la memoria y entrenar el cerebro de forma interactiva y amena. 
Estos juegos están divididos en función del área a entrenar, tal y como se busca con nuestro proyecto.

\begin{figure}[H]
    \centering
    \includegraphics[width=0.5\linewidth]{imgs/app_lumosity.png}
    \caption{Lumosity - Entrenador Cerebral}
\end{figure}

A pesar de no estar diseñada específicamente para el TDAH, trabaja áreas como la memoria de trabajo y la planificación, aspectos clave a mejorar en el caso de las personas con TDAH, ya que están directamente relacionados con su capacidad para organizar tareas, seguir instrucciones y mantener la información relevante en mente durante la realización de una actividad. 
En este sentido, algunos juegos de la plataforma \textit{Lumosity} como \textit{Memory Matrix} estimulan la retención activa de patrones o secuencias visuales, reforzando la memoria de trabajo. 
Para ello, el usuario tiene que memorizar qué celdas de una matriz están coloreadas para, posteriormente, seleccionarlas.

\begin{figure}[H]
    \centering
    \includegraphics[width=0.5\linewidth]{imgs/app_lumosity_mm.jpg}
    \caption{Minijuego Memory Matrix}
\end{figure}

Por otro lado, juegos como Pirate Passage, que requiere planificar rutas óptimas para alcanzar un objetivo, fomentan habilidades de planificación y toma de decisiones. 

\begin{figure}[H]
    \centering
    \includegraphics[width=0.5\linewidth]{imgs/app_lumosity_ps.png}
    \caption{Minijuego Pirate Passage}
\end{figure}

Además, contiene una página con estadísticas que muestra el desempeño del usuario, lo cual puede resultar muy útil para personas con TDAH, ya que de este modo, pueden recibir retroalimentación acerca de sus logros y aspectos a mejorar. 

Sin embargo, aunque estos juegos se adaptan al nivel de los usuarios, algunos de ellos pueden llegar a resultar algo complejos cuando el usuario va avanzando, pues están realizados para un intervalo de edad mayor. 
Nuestra plataforma, por el contrario, estará centrada en un público más joven e inexperto, por lo que los juegos que se buscan tendrán una dificultad algo menor.
\newpage

\subsubsection{Cognifit - Test y Juegos}
Esta tercera aplicación \parencite{cognifit} es una plataforma de estimulación cognitiva basada en la neuropsicología y en la evaluación del rendimiento, permitiendo tanto la estimulación como el seguimiento personalizado.

\begin{figure}[H]
    \centering
    \includegraphics[width=0.5\linewidth]{imgs/app_cognifit.jpg}
    \caption{Cognifit - Test y Juegos}
\end{figure}

Una de sus principales fortalezas es la posibilidad de crear programas adaptados a perfiles específicos como el TDAH. 
Para la creación de perfiles, el usuario es expuesto a una serie de preguntas sobre sus hábitos, estilo de vida y dificultades cognitivas; obteniendo un perfil cognitivo personalizado.
Además, incluye juegos interactivos diseñados para fortalecer áreas cognitivas comúnmente afectadas en este trastorno, como la memoria de trabajo, la atención sostenida o la planificación.

Esta aplicación destaca por su gran cantidad de artículos científicos que avalan su eficacia en la mejora de diversas funciones cognitivas, mejorando significativamente la memoria de trabajo, realizando un gran entrenamiento cognitivo interactivo y gamificado; y evaluando y adaptando los ejercicios a nivel de usuario \parencite{shatil2014}.
Por ejemplo, para la memoria de trabajo, \textit{CogniFit} incluye tareas que requieren retener y manipular información por breves períodos, así como vídeos explicativos para cada área, lo que resulta útil en intervenciones orientadas a la mejora del rendimiento académico. 
A continuación, se muestra un minijuego en el que la plataforma muestra un número al usuario durante un segundo y este tiene que recordarlo y completar la ecuación disparando al objeto correcto.

\begin{figure}[H]
    \centering
    \includegraphics[width=0.5\linewidth]{imgs/app_cognifit_memory.jpg}
    \caption{Minijuego de memoria}
\end{figure}

Además, la plataforma puede ser utilizada tanto en entornos clínicos como en el hogar o el contexto escolar, y permite la supervisión por parte de profesionales, educadores o familias.

En definitiva, \textit{CogniFit} representa una herramienta digital útil para trabajar habilidades cognitivas específicas de forma lúdica y motivadora, algo destacable para el TDAH, donde el componente motivacional y la retroalimentación inmediata son fundamentales.

\subsection{Comparativa de aplicaciones}
Tras el análisis de otras aplicaciones, se ha realizado un estudio de aspectos clave como las competencias trabajadas en cada una de estas aplicaciones, el precio de uso,  el sistema operativo requerido o la edad a la que está dirigida. Con esta información, se han obtenido las siguientes tablas resumen:

\begin{table}[H]
\centering
\begin{tabularx}{\textwidth}{|X|X|X|X|}
    \hline
    \textbf{Competencia} & \textbf{Respira, Piensa, Actúa} & \textbf{Lumosity} & \textbf{CogniFit}\\
    \hline
    Autorregulación emocional         & x &   &   \\
    \hline
    Resolución de conflictos sociales & x &   &   \\
    \hline
    Identificación emocional          & x &   &   \\
    \hline
    Memoria de trabajo                &   & x & x \\
    \hline
    Organización de objetos/tareas    &   &   & x \\
    \hline
    Planificación / anticipación      &   & x & x \\
    \hline
    Control inhibitorio / impulsos    & x & x & x \\
    \hline
    Gestión del tiempo                &   & x &   \\
    \hline
    Motivación por refuerzo           & x & x & x \\
    \hline
\end{tabularx}
\caption{Comparativa entre competencias de aplicaciones}
\end{table}

\begin{table}[H]
\centering
\begin{tabularx}{\textwidth}{|X|X|X|X|}
    \hline
    \textbf{Característica} & \textbf{Respira, Piensa, Actúa} & \textbf{Lumosity} & \textbf{CogniFit} \\
    \hline
    Precio & Gratuita & Gratis con limitación; suscripción premium para escoger juegos & Suscripción obligatoria \\
    \hline
    Sistema Operativo & iOS, Android & Web, iOS, Android & Web, iOS, Android \\
    \hline
    Edad recomendada & 2–5 años & $\ge$ 13 años & 9–17 años \\
    \hline
    Nivel de dificultad & Fijo & Adaptativo & Adaptativo \\
    \hline
    Perfil TDAH & No & No & Sí \\
    \hline
    Personalización & No & Sí & Sí \\
    \hline
    Estudios que evalúan su eficacia & No & Sí & Sí \\
    \hline
    Registro de usuarios y perfiles & No & Sí & Sí \\
    \hline
    Seguimiento del progreso & No & Sí & Sí \\
    \hline
    Refuerzos y recompensas & No & Sí & Sí \\
    \hline
\end{tabularx}
\caption{Comparativa de las aplicaciones según características}
\end{table}

\subsection{Artículos científicos relacionados}
Para el diseño de una aplicación dirigida a niños y niñas con TDAH, resulta esencial revisar la evidencia científica disponible sobre el diseño y la eficacia de las herramientas tecnológicas en este ámbito. 
Así pues, se destacan dos artículos recientes que ofrecen claros resultados positivos ante las intervenciones tecnológicas sobre distintas funciones cognitivas y conductuales afectadas por el TDAH.

En primer lugar, encontramos el artículo \textit{Meta-analysis of the efficacy of digital therapies in children with attention-deficit hyperactivity disorder} \parencite{tech_interventions_tdah}, cuyo objetivo es la evaluación de la efectividad de intervenciones digitales para mejorar los síntomas del TDAH en estudiantes. 
Este estudio sigue las directrices PRISMA y el Manual Cochrane \parencite{prisma}, lo cual indica que la revisión es sistemática y fiable; además de estar registrado en PROSPERO \parencite{prospero}, lo que añade también transparencia al proceso.

Para la evaluación se utilizan instrumentos que podrían ser útiles para la validación de la plataforma. 
Por un lado, se utiliza la \textit{Escala de Calificación del Trastorno por Déficit de Atención e Hiperactividad (ADHD-RS)} para medir la inatención e impulsividad del alumnado \parencite{adhdrs}. 
Para ello, se basa en los criterios del \textit{Manual Diagnóstico y Estadístico de los Trastornos Mentales, Cuarta Edición (DSM-IV)}, el cual contiene descripciones detalladas de los diferentes trastornos mentales, incluyendo criterios diagnósticos específicos para cada uno \parencite{dsmiv}. 
Por otro lado, también se utiliza la herramienta \textit{Evaluación Conductual de la Función Ejecutiva-2 (BRIEF-2)}, un cuestionario completado por padres o tutores que mide las funciones ejecutivas, evaluando los aspectos más cotidianos y conductuales \parencite{brief2}.

Además de los resultados sobre eficacia, este metaanálisis describe las características de las terapias digitales incluidas, que abarcan fundamentalmente programas de entrenamiento cognitivo asistidos por ordenador y videojuegos terapéuticos. 
Estas intervenciones se orientan a mejorar habilidades como la atención sostenida, la memoria de trabajo o la autorregulación conductual. 
Una de sus características principales es el uso de mecánicas de gamificación (niveles progresivos, retroalimentación inmediata y recompensas), lo que favorece la motivación y la adherencia en la población infantil con TDAH.

En segundo lugar, encontramos el artículo \textit{Effectiveness of Technology-Based Interventions for School-Age Children With Attention-Deficit/Hyperactivity Disorder: Systematic Review and Meta-Analysis of Randomized Controlled Trials} \parencite{meta_digital_adhd}, cuyo objetivo es la realización de una revisión sistemática de las intervenciones tecnológicas para el alumnado con TDAH, así como un metaanálisis de los resultados tras estas intervenciones. 
Este estudio también está registrado en PROSPERO y sigue las directrices PRISMA para garantizar el rigor metodológico.

En este segundo artículo, se utilizaron herramientas de evaluación como las escalas de calificación, los cuestionarios o las pruebas estandarizadas relacionadas con síntomas de inatención e impulsividad. 
Aunque el artículo no detalla los nombres exactos de las herramientas utilizadas en cada ECA, sí especifica que las medidas empleadas evaluaban aspectos como inhibición, memoria de trabajo, atención, planificación, metacognición y calidad de vida. 
Esta metodología puede servir como referencia para la selección de instrumentos validados en futuras investigaciones, como por ejemplo el ADHD-RS o el BRIEF-2, ya utilizados en otros estudios similares.

En este caso, se analizan distintos tipos de intervenciones tecnológicas aplicadas en ensayos clínicos aleatorizados: programas de entrenamiento cognitivo digital, neurofeedback, aplicaciones educativas interactivas y entornos de realidad virtual. 
Estas herramientas trabajan un abanico de competencias relacionadas con el TDAH, entre ellas la atención visual, el control de impulsos, la memoria de trabajo y la planificación. 
El estudio destaca el uso de plataformas gamificadas o entornos lúdicos, lo que potencia la implicación del alumnado y contribuye a la mejora en medidas clínicas y funcionales.