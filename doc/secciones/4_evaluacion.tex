\section{Evaluación con usuarios}
Para evaluar la efectividad y usabilidad de los minijuegos desarrollados, se llevaron a cabo diversas evaluaciones con usuarios reales.
Estas evaluaciones se centraron en medir el grado de aceptabilidad y usabilidad de los juegos, así como en recoger feedback para futuras mejoras.
Para ello, se realizaron pruebas tanto con un grupo no clínico de estudiantes de bachillerato como con un grupo clínico de niños y niñas diagnosticados con TDAH.
Como hipótesis inicial, se planteó que los juegos serían bien aceptados por ambos grupos, pudiendo utilizarlos de manera sencilla e intuitiva.

En estas pruebas se les pidió a los participantes que jugaran a un nivel de cada uno de los minijuegos desarrollados.
Durante la sesión, se recogieron datos sobre el uso de los juegos: el tiempo empleado para cada nivel, los errores cometidos, las dudas planteadas, las estrategias utilizadas por los jugadores y jugadoras y, finalmente, los resultados obtenidos.

De cara a la retroalimentación, se planteó el desarrollo tanto de un cuestionario SUS (System Usability Scale) como de un cuestionario TAM (Technology Acceptance Model), ambos adaptados al contexto de los juegos serios.
Por un lado, el cuestionario SUS se suministra a los usuarios que utilizan la plataforma (con una experiencia directa), idealmente inmediatamente después de utilizarla para que puedan evaluar su experiencia de uso de manera inmediata y precisa \parencite{sus}.
Cada pregunta se puntúa en una escala likert del 1 al 5, siendo 1 \textit{'totalmente en desacuerdo'} y 5 \textit{'totalmente de acuerdo'}, alternando preguntas positivas con negativas.
Por otro lado, el cuestionario TAM facilita la comprensión y evaluación de la aceptación de los usuarios hacia las nuevas tecnologías, permitiendo desarrollar e implementar mejores sistemas.
Se ha probado en muchas investigaciones, ante diversos contextos y ha demostrado ser una herramienta confiable para conocer la aceptación de estas tecnologías \parencite{tam}.

Inicialmente, se planteó el uso de ambos, pero tras la revisión de ambos con detenimiento, se observó que muchas de las preguntas del cuestionario SUS se solapaban con las del cuestionario TAM, llegando a ser redundantes.
A esto se le sumó la observación de la corta duración de las sesiones, por lo que se optó por realizar un único cuestionario TAM \cref{fig:tam}.
El cuestionario incluyó las siguientes dimensiones:

\newpage
\begin{itemize}
    \item \textbf{F} - Facilidad de uso
    \item \textbf{DP} - Disfrute percibido
    \item \textbf{IU} - Intención de uso
    \item \textbf{C} - Compatibilidad con la rutina
    \item \textbf{AE} - Autoeficacia percibida
    \item \textbf{AC} - Aceptación global
\end{itemize}

Cada ítem se respondió mediante una escala Likert de 1 a 5, acompañada de iconos emocionales para mejorar la comprensión según la edad del usuario. 
Además, se incluyeron dos preguntas abiertas sobre errores detectados y sugerencias de mejora.

\begin{figure}[H]
    \centering
    \includegraphics[width=1\linewidth]{imgs/tam.png}
    \caption{Cuestionario TAM}
    \label{fig:tam}
\end{figure}

\newpage
\subsection{Evaluación con grupo no clínico}
La evaluación con el grupo no clínico tenía como objetivo medir la aceptabilidad, facilidad de uso e intención de uso en población escolar general.
Además, se buscaba la detección de posibles errores o dificultades en la interacción con los minijuegos, recogiendo posibles sugerencias de mejoras.
Finalmente, se trató de evaluar también la compatibilidad con la rutina con la rutina escolar de estos estudiantes.

Durante la evaluación, se contó con la participación de únicamente 6 estudiantes de bachillerato, todos ellos con 16 años.
Estas pruebas se llevaron a cabo en un entorno controlado, en una sala de la Facultad de Ciencias de la Salud de la Universidad de Granada.
A pesar del tamaño de la muestra, se pudo obtener información valiosa sobre la usabilidad y aceptación de los minijuegos.

\subsubsection{Resultados en los minijuegos}
Tras analizar los datos recogidos durante las sesiones de juego, en general, se observó que los participantes completaron los niveles de los minijuegos con éxito, mostrando un buen nivel de comprensión de las mecánicas de juego.
No obstante, se identificaron algunas áreas de mejora, las cuales se detallarán posteriormente.

Respecto a los puntos obtenidos en cada minijuego, se observó que los participantes lograron puntajes altos en la mayoría de los niveles, lo que indica que, generalmente, las mecánicas de juego eran accesibles y comprensibles.
Sin embargo, podemos observar que los resultados obtenidos en el juego \textit{Detective emociones} fueron notablemente más bajos en comparación con los otros tres minijuegos.
Esto podría deberse a que, para una situación de juego única, los participantes pueden manifestar emociones distintas, lo que aumentó la complejidad del juego en comparación con los otros minijuegos, donde las tareas eran más directas y específicas.

En definitiva, se obtuvo una media de puntuación de 2.25 sobre 3 en los minijuegos, lo que indica un buen nivel de desempeño general por parte de los participantes.

\begin{figure}[H]
    \centering
    \includegraphics[width=1\linewidth]{imgs/ncg_points_table.png}
    \caption{Puntuaciones en los minijuegos - Tabla}
    \label{fig:ncg_points_table}
\end{figure}

\begin{figure}[H]
    \centering
    \includegraphics[width=1\linewidth]{imgs/ncg_points_graphic.png}
    \caption{Puntuaciones en los minijuegos - Gráfico}
    \label{fig:ncg_points_graph}
\end{figure}

\newpage
Entre los errores cometidos, se observó que la mayoría de los participantes cometieron pocos errores en los niveles de los minijuegos.
No obstante, se pudieron identificar ciertos problemas específicos en algunos minijuegos:
\begin{itemize}
    \item En \textit{Metro de la memoria}, 3 de los 6 participantes tuvieron dificultades debido a que no recordaban correctamente la secuencia o, simplemente, no la habían leído bien.
    Esto sugiere que podría ser beneficioso incluir una opción para repetir la secuencia o proporcionar pistas adicionales.
    De esta forma, se facilitaría la experiencia de juego y se reduciría la frustración de los jugadores.
    \item En \textit{Desván mágico}, un participante falló al organizar los objetos correctamente, ya que trató de almacenarlos todos en algún cajón.
    Esto indica que, posiblemente, no leyó bien las instrucciones o no comprendió completamente la mecánica del juego.
    \item En \textit{Detective emociones}, ninguno de los participantes obtuvo la puntuación máxima.
    Esto sugiere la necesidad de plantear la posibilidad de incluir varias respuestas correctas para cada situación, ya que las emociones pueden interpretarse de diferentes maneras.
    \item En \textit{Semáforo emocional}, no se observaron errores significativos, ya que todos los participantes, a excepción de uno, lograron completar el nivel con éxito.    
\end{itemize}

Respecto al tiempo empleado en completar cada nivel, se observó que los participantes tardaron tiempos muy parecidos en todos los minijuegos (entre 30 y 60 segundos aproximadamente) salvo, una vez más, en \textit{Detective emociones}, donde el tiempo medio fue considerablemente mayor (llegando a alcanzar los 3 minutos y medio).
Esto se debió a que los participantes tuvieron que reflexionar más sobre las respuestas correctas, ya que las emociones pueden ser subjetivas y variar según la interpretación individual.

\begin{figure}[H]
    \centering
    \includegraphics[width=1\linewidth]{imgs/ncg_time_table.png}
    \caption{Tiempo empleado en los minijuegos - Tabla}
    \label{fig:ncg_time_table}
\end{figure}

\begin{figure}[H]
    \centering
    \includegraphics[width=1\linewidth]{imgs/ncg_time_graphic.png}
    \caption{Tiempo empleado en los minijuegos - Gráfico}
    \label{fig:ncg_time_graphic}
\end{figure}

En conclusión, los resultados obtenidos en los minijuegos por parte del grupo no clínico indican que, en general, los participantes pudieron comprender y completar las tareas propuestas.
No obstante, se identificaron áreas de mejora específicas en algunos minijuegos, lo que sugiere la necesidad de realizar ajustes para optimizar la experiencia de juego y facilitar la comprensión de las mecánicas.
Podemos destacar los siguientes cambios:

\begin{itemize}
    \item Incorporación de un botón de ayuda en el juego \textit{Metro de la memoria}.
    \item Incorporación de mapas más complejos en los niveles avanzados del juego \textit{Metro de la memoria}.
    \item Incorporación de más cajones y objetos en el juego \textit{Desván mágico}.
    \item Inclusión de una batería de objetos para cada nivel, mostrando diferentes objetos en cada intento en el juego \textit{Desván mágico}.
    \item Posibilidad de varias respuestas correctas en el juego \textit{Detective emociones}.
    \item Almacenamiento del tiempo de juego e incorporación de un alarma si se supera un límite establecido (30 minutos aproximadamente).
    \item Posibilidad de elección del tipo de contraseña (texto o imágenes) tanto en el registro como en el inicio de sesión.
    \item Posibilidad de asignación de distractores por parte de los tutores en los juegos.
\end{itemize}

Gran parte de estas mejoras serán implementadas en futuras versiones de los minijuegos, con el objetivo de optimizar la experiencia del usuario y aumentar la efectividad de los juegos como herramientas de apoyo para niños y niñas con TDAH.

\subsubsection{Resultados del cuestionario TAM}
Tras analizar las respuestas del cuestionario TAM, las cuales fueron 4 de los 6 estudiantes que realizaron las pruebas, se obtuvieron los siguientes resultados en cada una de las dimensiones evaluadas:

\begin{itemize}
    \item \textbf{Facilidad de uso (F)}: Los participantes consideraron que los minijuegos eran fáciles de usar, con una puntuación media de 4.06 sobre 5.
    \item \textbf{Disfrute percibido (DP)}: Los estudiantes expresaron un alto nivel de disfrute al jugar, con una puntuación media de 4.25 sobre 5.
    \item \textbf{Intención de uso (IU)}: La intención de utilizar los minijuegos en el futuro fue bastante más baja que los anteriores, con una puntuación media de 2.63 sobre 5.
    \item \textbf{Compatibilidad con la rutina (C)}: Los participantes consideraron que los minijuegos eran realtivamente compatibles con su rutina diaria, obteniendo una puntuación media de 3.38 sobre 5.
    \item \textbf{Autoeficacia percibida (AE)}: Los estudiantes se sintieron bastante seguros de su capacidad para utilizar los minijuegos, con una puntuación media de 3.67 sobre 5.
    \item \textbf{Aceptación global (AC)}: La aceptación general de los minijuegos fue alta, con una puntuación media de 3.88 sobre 5.
\end{itemize}

Por otro lado, en las preguntas abiertas, los participantes indicaron que no habían encontrado ningún error significativo durante su experiencia de juego.
Sin embargo, uno de los estudiantes sugirió como mejora el aumento de la dificultad en algunos niveles, para mantener el interés y el desafío a medida que avanzaban en el juego.

\newpage
En resumen, los resultados del cuestionario TAM indican que los minijuegos fueron bien recibidos por el grupo no clínico, con altas puntuaciones en facilidad de uso, disfrute percibido y aceptación global.
No obstante, la intención de uso fue relativamente baja, lo que este público podría no estar tan interesado en utilizar estos juegos en su rutina diaria.
Estos resultados proporcionan información valiosa para futuras mejoras y ajustes en los minijuegos, con el fin de optimizar la experiencia del usuario y aumentar la intención de uso.