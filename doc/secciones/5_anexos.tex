\section{Anexos}
\subsection{Cuestionario TAM de evaluación de LudiMind}
\subsubsection{Introducción}
Este cuestionario tiene como objetivo conocer tu opinión sobre el uso de LudiMind. No hay respuestas correctas o incorrectas: lo importante es lo que tú piensas y sientes.
Tus respuestas serán confidenciales y se utilizarán únicamente para mejorar la experiencia de uso y adaptar mejor las herramientas a tus necesidades. 
Por favor, responde con sinceridad a cada afirmación seleccionando la opción que mejor refleje tu grado de acuerdo. Usamos una escala de 5 puntos:

\begin{enumerate}
    \item Totalmente en desacuerdo
    \item En desacuerdo
    \item Ni de acuerdo ni en desacuerdo
    \item De acuerdo
    \item Totalmente de acuerdo
\end{enumerate}

\subsubsection{Preguntas de carácter general}

\begin{itemize}
    \item ¿Cuál es tu edad? 
    \item ¿Cuál es tu genero? 
    Hombre, mujer, prefiero no decirlo 
    \item ¿Cuál es tu rol en el proceso? 
    Estudiante, profesor, padre/madre, terapeuta
\end{itemize}

\subsubsection{Preguntas del cuestionario TAM}
\begin{table}[H]
\centering
\begin{tabularx}{\textwidth}{|c|X|}
\hline
\textbf{Código} & \textbf{Afirmación} \\
\hline
F1 & La forma de usar LudiMind es clara y comprensible \\
\hline
F2 & La forma de usar LudiMind es intuitiva \\
\hline
F3 & El uso de LudiMind no requiere mucho esfuerzo mental ni conocimientos tecnológicos \\
\hline
F4 & Pienso que es fácil de usar LudiMind \\
\hline
F5 & Es fácil aprender a usar LudiMind \\
\hline
F6 & La forma de uso es fácil y comprensible después de la fase de aprendizaje \\
\hline
F7 & Pienso que LudiMind se puede usar sin ayuda técnica \\
\hline
F8 & Valoro positivamente las ayudas y refuerzos auditivos y visuales de LudiMind \\
\hline
F9 & Cuando utilizo LudiMind, sé qué estoy haciendo en cada momento \\
\hline
DP1 & Es divertido o agradable usar las herramientas y juegos disponibles \\
\hline
DP2 & LudiMind incluye actividades adaptadas a los intereses de cada persona \\
\hline
IU1 & Tengo intención de usar LudiMind \\
\hline
IU2 & Uso juegos similares a LudiMind con frecuencia \\
\hline
C1 & LudiMind es compatible con el trabajo en el centro educativo o en casa \\
\hline
C2 & El uso de LudiMind encaja en la rutina diaria \\
\hline
AE1 & Tengo confianza en que puedo aprender a usar LudiMind fácilmente \\
\hline
AE2 & Me siento seguro/a usando LudiMind \\
\hline
AE3 & Tengo las habilidades necesarias para utilizar LudiMind \\
\hline
AC1 & Utilizar LudiMind para enseñar habilidades de organización y planificación es una buena idea \\
\hline
AC2 & Valoro positivamente herramientas como LudiMind \\
\hline
\end{tabularx}
\caption{Cuestionario TAM de evaluación de LudiMind}
\label{tab:tam_ludimind}
\end{table}

\subsubsection{Preguntas abiertas}
\begin{itemize}
    \item ¿Ha encontrado errores al utilizar los juegos? En caso afirmativo, indique cuáles.
    \item ¿Qué mejoras haría en los juegos?
\end{itemize}

\subsubsection{Consentimiento}
\begin{itemize}
    \item He leído la información proporcionada y doy mi consentimiento informado para participar en este estudio.
    \item Estoy de acuerdo en participar de manera voluntaria en la investigación.
    \item Autorizo el uso de mis datos de forma anónima y confidencial con fines exclusivamente de investigación.
\end{itemize}
