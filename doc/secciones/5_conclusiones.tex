\section{Conclusiones y trabajos futuros}
\subsection{Conclusiones}
Durante la realización de este proyecto, se han estudiado las características y dificultades con las que viven diariamente las personas con TDAH. 
Esto nos ha permitido desarrollar una aplicación lo más adaptada posible a estos usuarios, ayudando a mejorar sus habilidades cognitivas y emocionales a través de juegos educativos diseñados específicamente para sus necesidades.

Durante el desarrollo de la aplicación, he adquirido numerosos conocimientos sobre el desarrollo de software accesible, fortaleciendo mi capacidad de resolución de problemas y búsqueda de soluciones. 
Sin embargo, no solo he adquirido conocimientos técnicos, sino que también he aprendido sobre las necesidades y desafíos que se enfrentan las personas con TDAH en su vida diaria, tratando de solucionar gran parte de ellos y, en consecuencia, mejorar su día a día.

Una parte clave en este proyecto ha sido las pruebas con usuarios, las cuales han permitido obtener retroalimentación directa de los usuarios finales.
Además, esto hizo posible visualizar personalmente cómo los usuarios trabajan, pudiendo sacar conclusiones más exactas y que ayudaran a mejorar la aplicación.

En definitiva, LudiMind ha cumplido el objetivo principal por el que se planteó este proyecto: diseñar, de manera iterativa, una aplicación que, poco a poco, se fuera adaptando más al público objetivo, realizando pruebas que nos permitieran conocer si se había seguido el camino correcto o, por el contrario, debíamos cambiar el planteamiento de algunas de las funcionalidades. 

De este modo, este proyecto serviría como base para el desarrollo de la aplicación final, realizando más pruebas e incorporando todas las posibles mejoras que se pudieran encontrar, con el objetivo de conseguir la mejor versión posible de la aplicación.

\newpage
\subsection{Trabajos futuros}
A pesar de que LudiMind ha alcanzado un nivel considerable de funcionalidad y usabilidad, existen varias áreas en las que se podrían realizar mejoras y expansiones en el futuro:

\begin{itemize}
    \item \textbf{Implementación de misiones y recompensas:} Completar la funcionalidad de misiones y recompensas para motivar a los usuarios a seguir utilizando la aplicación y mejorar su experiencia.
    \item \textbf{Ampliación de juegos educativos:} Añadir aleatoriedad en los niveles de los juegos, consiguiendo así una mayor variedad y desafío para los usuarios.
    \item \textbf{Tutoriales interactivos:} Desarrollar tutoriales interactivos para cada juego, facilitando a los usuarios la comprensión de las reglas y objetivos de cada actividad.
    \item \textbf{Plataforma para educadores, padres y madres:} Desarrollar una sección en la aplicación que permita a educadores y padres supervisar el progreso de los usuarios y personalizar las actividades según sus necesidades.
\end{itemize}

Así, LudiMind podría seguir evolucionando y adaptándose a las necesidades de sus usuarios, proporcionando una herramienta cada vez más efectiva para mejorar las habilidades cognitivas y emocionales de las personas con TDAH.