\section{Mecánica de los juegos}

\subsection{Metro de la Memoria}

\subsubsection{Descripción del juego}
Este minijuego está diseñado para ayudar a los estudiantes con TDAH a mejorar su memoria y concentración. 
El juego se desarrolla en un mapa ficticio que representa una red de metro, con diferentes estaciones y líneas de tren con nombres llamativos y divertidos. 
El jugador o jugadora debe llevar a un personaje desde un punto A hasta un punto B, cambiando de línea en las estaciones adecuadas.
Para hacerlo, el juego le enseña frases mnemotécnicas que resumen el trayecto.

\subsubsection{Funciones cognitivas trabajadas}
Este juego trabaja principalmente las siguientes funciones cognitivas:
\begin{itemize}
    \item \textbf{Memoria de trabajo:} Los jugadores y jugadoras deben mantener en mente la información sobre las estaciones de entrada y salida, así como la frase proporcionada a modo de pista.
    \item \textbf{Flexibilidad cognitiva:} El juego requiere que los jugadores y jugadoras vayan cambiando de línea en las estaciones correctas, lo que implica adaptarse a nuevas reglas y situaciones.
\end{itemize}

\subsubsection{Desarrollo del juego}
El juego cuenta con varios niveles, cada uno con un nivel de dificultad creciente. 
En las primeras fases, se presentan frases muy concretas, especificando gran parte de las paradas por las que el usuario debe pasar,
mientras que en las fases posteriores se introducen frases más complejas. 
A medida que avanzan en el juego, estas frases se vuelven más abstractas, requiriendo que los jugadores y jugadoras utilicen su memoria de trabajo y flexibilidad cognitiva para recordar y aplicar la información.

Inicialmente, se muestra a un avatar que le plantea al jugador o jugadora el reto del nivel, indicando la estación de inicio y la estación de destino. 

\begin{figure}[H]
    \centering
    \includegraphics[width=1\linewidth]{imgs/desarrollo_metro_1.png}
    \caption{Metro de la Memoria - Inicio del nivel}
\end{figure}

Luego, se presenta a modo de pista una frase mnemotécnica que resume el trayecto que el usuario debe seguir.

\begin{figure}[H]
    \centering
    \includegraphics[width=1\linewidth]{imgs/desarrollo_metro_2.png}
    \caption{Metro de la Memoria - Frase mnemotécnica}
\end{figure}

A continuación, el jugador o jugadora debe seleccionar las estaciones correctas en el orden adecuado para completar el trayecto.
El usuario solo podrá avanzar si selecciona estaciones contiguas y, una vez seleccionadas todas, podrá resolver el  para comprobar su corrección.

\begin{figure}[H]
    \centering
    \includegraphics[width=1\linewidth]{imgs/desarrollo_metro_3.png}
    \includegraphics[width=1\linewidth]{imgs/desarrollo_metro_4.png}
    \caption{Metro de la Memoria - Selección de estaciones}
\end{figure}

Si el jugador o jugadora selecciona las estaciones correctas, recibirá una puntuación de tres estrellas, completando el nivel.
El usuario contará con tres intentos cada vez que inicie el juego, de forma que el número de estrellas obtenidas se calculará en función del número de intentos realizados.